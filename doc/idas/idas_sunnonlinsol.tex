%===============================================================================
\chapter{Description of the SUNNonlinearSolver module}\label{c:sunnonlinsol}
%===============================================================================
\index{SUNNonlinearSolver@\texttt{SUNNonlinearSolver} module}
{\sundials} time integration packages are written in terms of generic nonlinear
solver operations defined by the {\sunnonlinsol} API and implemented by a
particular {\sunnonlinsol} module of type \noindent\Id{SUNNonlinearSolver}.
Users can supply their own {\sunnonlinsol} module, or use one of the modules
provided with {\sundials}. Depending on the package, nonlinear solver modules
can either target system presented in a rootfinding ($F(y) = 0$) or fixed-point
($G(y) = y$) formulation. For more information on the formulation of the
nonlinear system(s) see section \ref{s:sunnonlinsol_interface}.

The time integrators in {\sundials} specify a default nonlinear solver module
and as such this chapter is intended for users that wish to use a non-default
nonlinear solver module or would like to provide their own nonlinear solver
implementation. Users interested in using a non-default solver module may skip
the description of the {\sunnonlinsol} API in section \ref{s:sunnonlinsol_api}
and proceeded to the subsequent sections in this chapter that describe the
{\sunnonlinsol} modules provided with {\sundials}.

For users interested in providing their own {\sunnonlinsol} module, the
following section presents the {\sunnonlinsol} API and its implementation
beginning with the definition of {\sunnonlinsol} functions in sections
\ref{ss:sunnonlinsol_corefn} -- \ref{ss:sunnonlinsol_getfn}. This is followed by
the definition of functions supplied to a nonlinear solver implementation in
section \ref{ss:sunnonlinsol_sunsuppliedfn}. A table of nonlinear solver return
codes is given in section \ref{ss:sunnonlinsol_returncodes}. The
\id{SUNNonlinearSolver} type and the generic {\sunnonlinsol} module are defined
in section \ref{ss:sunnonlinsol_generic}. Section \ref{ss:sunnonlinsol_sens}
describes how {\sunnonlinsol} models interface with {\sundials} integrators
providing sensitivity analysis capabilities ({\cvodes} and {\idas}). Finally,
section \ref{ss:sunnonlinsol_custom} lists the requirements for supplying a
custom {\sunnonlinsol} module. Users wishing to supply their own {\sunnonlinsol}
module are encouraged to use the {\sunnonlinsol} implementations provided with
{\sundials} as a template for supplying custom nonlinear solver
modules.


% ====================================================================
\section{The SUNNonlinearSolver API}
\label{s:sunnonlinsol_api}
% ====================================================================

The {\sunnonlinsol} API defines several nonlinear solver operations that enable
{\sundials} integrators to utilize any {\sunnonlinsol} implementation that
provides the required functions. These functions can be divided into three
categories. The first are the core nonlinear solver functions. The second group
of functions consists of set routines to supply the nonlinear solver with
functions provided by the {\sundials} time integrators and to modify solver
parameters. The final group consists of get routines for retrieving nonlinear
solver statistics. All of these functions are defined in the header file
\id{sundials/sundials\_nonlinearsolver.h}.

% ====================================================================
\subsection{SUNNonlinearSolver core functions}
\label{ss:sunnonlinsol_corefn}
% ====================================================================
The core nonlinear solver functions consist of two required functions to get the
nonlinear solver type (\id{SUNNonlinsSolGetType}) and solve the nonlinear system
(\id{SUNNonlinSolSolve}). The remaining three functions for nonlinear solver
initialization (\id{SUNNonlinSolInitialization}), setup\\ \noindent
(\id{SUNNonlinSolSetup}), and destruction (\id{SUNNonlinSolFree}) are optional.

\ucfunctionf{SUNNonlinSolGetType}
{
  type = SUNNonlinSolGetType(NLS);
}
{
  The \textit{required} function \ID{SUNNonlinSolGetType} returns
  nonlinear solver type.
}
{
  \begin{args}[NLS]
  \item[NLS] (\id{SUNNonlinearSolver})
    a {\sunnonlinsol} object.
  \end{args}
}
{
  The return value \id{type} (of type \id{int}) will be one of the
  following:
  \begin{args}[SUNNONLINEARSOLVER\_FIXEDPOINT]
  \item[\Id{SUNNONLINEARSOLVER\_ROOTFIND}]
    \id{0}, the {\sunnonlinsol} module solves $F(y) = 0$.
  \item[\Id{SUNNONLINEARSOLVER\_FIXEDPOINT}]
    \id{1}, the {\sunnonlinsol} module solves $G(y) = y$.
  \end{args}
}
{}
% --------------------------------------------------------------------
\ucfunctionf{SUNNonlinSolInitialize}
{
  retval = SUNNonlinSolInitialize(NLS);
}
{
  The \textit{optional} function \ID{SUNNonlinSolInitialize} performs
  nonlinear solver initialization and may perform any necessary memory
  allocations.
}
{
  \begin{args}[NLS]
  \item[NLS] (\id{SUNNonlinearSolver})
    a {\sunnonlinsol} object.
  \end{args}
}
{
  The return value \id{retval} (of type \id{int}) is zero for a
  successful call and a negative value for a failure.
}
{
  It is assumed all solver-specific options have been set prior to
  calling \\ \noindent
  \id{SUNNonlinSolInitialize}. {\sunnonlinsol} implementations
  that do not require initialization may set this operation
  to \id{NULL}.
}
% --------------------------------------------------------------------
\ucfunctionf{SUNNonlinSolSetup}
{
  retval = SUNNonlinSolSetup(NLS, y, mem);
}
{
  The \textit{optional} function \ID{SUNNonlinSolSetup} performs any
  solver setup needed for a nonlinear solve.
}
{
  \begin{args}[NLS]
  \item[NLS] (\id{SUNNonlinearSolver})
    a {\sunnonlinsol} object.
  \item[y] (\id{N\_Vector})
    the initial iteration passed to the nonlinear solver.
  \item[mem] (\id{void *})
    the {\sundials} integrator memory structure.
  \end{args}
}
{
  The return value \id{retval} (of type \id{int}) is zero for a
  successful call and a negative value for a failure.
}
{
  {\sundials} integrators call \id{SUNonlinSolSetup} before each step
  attempt. {\sunnonlinsol} implementations that do not require setup  may set
  this operation to \id{NULL}.
}
% --------------------------------------------------------------------
\ucfunctionf{SUNNonlinSolSolve}
{
  retval = SUNNonlinSolSolve(NLS, y0, ycor, w, tol, callLSetup, mem);
}
{
  The \textit{required} function \ID{SUNNonlinSolSolve} solves the
  nonlinear system $F(y)=0$ or $G(y)=y$.
}
{
  \begin{args}[callLSetup]
  \item[NLS] (\id{SUNNonlinearSolver})
    a {\sunnonlinsol} object.
  \item[y0] (\id{N\_Vector})
    the predicted value for the new solution state. This \textit{must}
    remain unchanged throughout the solution process. See section
    \ref{s:sunnonlinsol_interface} for more detail on the nonlinear system
    formulation.
  \item[ycor] (\id{N\_Vector})
    on input the initial guess for the correction to the predicted state (zero)
    and on output the final correction to the predicted state. See section
    \ref{s:sunnonlinsol_interface} for more detail on the nonlinear system
    formulation.
  \item[w] (\id{N\_Vector})
    the solution error weight vector used for computing weighted error norms.
  \item[tol] (\id{realtype})
    the requested solution tolerance in the weighted root-mean-squared norm.
  \item[callLSetup] (\id{booleantype})
    a flag indicating that the integrator recommends for the linear
    solver setup function to be called.
  \item[mem] (\id{void *})
    the {\sundials} integrator memory structure.
  \end{args}
}
{
  The return value \id{retval} (of type \id{int}) is zero for a successul solve,
  a positive value for a recoverable error (i.e., the solve failed and the
  integrator should reduce the step size and reattempt the step), and a negative
  value for an unrecoverable error (i.e., the solve failed and the integrator
  should halt and return an error to the user).
}
{}
% --------------------------------------------------------------------
\ucfunctionf{SUNNonlinSolFree}
{
  retval = SUNNonlinSolFree(NLS);
}
{
  The \textit{optional} function \ID{SUNNonlinSolFree} frees any
  memory allocated by the nonlinear solver.
}
{
  \begin{args}[NLS]
  \item[NLS] (\id{SUNNonlinearSolver})
    a {\sunnonlinsol} object.
  \end{args}
}
{
  The return value \id{retval} (of type \id{int}) should be zero for a
  successful call, and a negative value for a failure. {\sunnonlinsol}
  implementations that do not allocate data may set this operation
  to \id{NULL}.
}
{}


% ====================================================================
\subsection{SUNNonlinearSolver set functions}
\label{ss:sunnonlinsol_setfn}
% ====================================================================
The following set functions are used to supply nonlinear solver modules with
functions defined by the {\sundials} integrators and to modify solver
parameters. Only the routine for setting the nonlinear system defining function
(\id{SUNNonlinSolSetSysFn} is required. All other set functions are optional.

\ucfunctionf{SUNNonlinSolSetSysFn}
{
  retval = SUNNonlinSolSetSysFn(NLS, SysFn);
}
{
  The \textit{required} function \ID{SUNNonlinSolSetSysFn} is used
  to provide the nonlinear solver with the function defining the
  nonlinear system. This is the function $F(y)$ in $F(y)=0$ for
  \id{SUNNONLINEARSOLVER\_ROOTFIND} modules or $G(y)$ in $G(y)=y$ for\\
  \id{SUNNONLINEARSOLVER\_FIXEDPOINT} modules.
}
{
  \begin{args}[SysFn]
  \item[NLS] (\id{SUNNonlinearSolver})
    a {\sunnonlinsol} object.
  \item[SysFn] (\id{SUNNonlinSolSysFn})
    the function defining the nonlinear system. See section
    \ref{ss:sunnonlinsol_sunsuppliedfn} for the definition of
    \id{SUNNonlinSolSysFn}.
  \end{args}
}
{
  The return value \id{retval} (of type \id{int}) should be zero for a
  successful call, and a negative value for a failure.
}
{}
% --------------------------------------------------------------------
\ucfunctionf{SUNNonlinSolSetLSetupFn}
{
  retval = SUNNonlinSolSetLSetupFn(NLS, LSetupFn);
}
{
  The \textit{optional} function \ID{SUNNonlinSolLSetupFn} is called
  by {\sundials} integrators to provide the nonlinear solver with
  access to its linear solver setup function.
}
{
  \begin{args}[LSetupFn]
  \item[NLS] (\id{SUNNonlinearSolver})
    a {\sunnonlinsol} object.
  \item[LSetupFn] (\id{SUNNonlinSolLSetupFn})
    a wrapper function to the {\sundials} integrator's linear solver setup
    function. See section \ref{ss:sunnonlinsol_sunsuppliedfn} for the
    definition of \\ \noindent
    \id{SUNNonlinLSetupFn}.
  \end{args}
}
{
  The return value \id{retval} (of type \id{int}) should be zero for a
  successful call, and a negative value for a failure.
}
{
  The \id{SUNNonlinLSetupFn} function sets up the linear system $Ax=b$ where
  $A = \frac{\partial F}{\partial y}$ is the linearization of the nonlinear
  residual function $F(y) = 0$ (when using {\sunlinsol} direct linear solvers)
  or calls the user-defined preconditioner setup function (when using
  {\sunlinsol} iterative linear solvers). {\sunnonlinsol} implementations that
  do not require solving this system, do not utilize {\sunlinsol} linear
  solvers, or use {\sunlinsol} linear solvers that do not require setup may set
  this operation to \id{NULL}.
}
% --------------------------------------------------------------------
\ucfunctionf{SUNNonlinSolSetLSolveFn}
{
  retval = SUNNonlinSolSetLSolveFn(NLS, LSolveFn);
}
{
  The \textit{optional} function \ID{SUNNonlinSolSetLSolveFn} is called by
  {\sundials} integrators to provide the nonlinear solver with access to
  its linear solver solve function.
}
{
  \begin{args}[LSolveFn]
  \item[NLS] (\id{SUNNonlinearSolver})
    a {\sunnonlinsol} object
  \item[LSolveFn] (\id{SUNNonlinSolLSolveFn})
    a wrapper function to the {\sundials} integrator's linear solver solve
    function. See section \ref{ss:sunnonlinsol_sunsuppliedfn} for the definition
    of\\ \noindent
    \id{SUNNonlinSolLSolveFn}.
  \end{args}
}
{
  The return value \id{retval} (of type \id{int}) should be zero for a
  successful call, and a negative value for a failure.
}
{
  The \id{SUNNonlinLSolveFn} function solves the linear system $Ax=b$ where
  $A = \frac{\partial F}{\partial y}$ is the linearization of the nonlinear
  residual function $F(y) = 0$. {\sunnonlinsol} implementations that do not
  require solving this system or do not use {\sunlinsol} linear solvers may set
  this operation to \id{NULL}.
}
% --------------------------------------------------------------------
\ucfunctionf{SUNNonlinSolSetConvTestFn}
{
  retval = SUNNonlinSolSetConvTestFn(NLS, CTestFn, ctest\_data);
}
{
  The \textit{optional} function \ID{SUNNonlinSolSetConvTestFn} is used to
  provide the nonlinear solver with a function for determining if the nonlinear
  solver iteration has converged. This is typically called by {\sundials}
  integrators to define their nonlinear convergence criteria, but may be replaced
  by the user.
}
{
  \begin{args}[ctest\_data]
  \item[NLS] (\id{SUNNonlinearSolver})
    a {\sunnonlinsol} object.
  \item[CTestFn] (\id{SUNNonlineSolConvTestFn})
    a {\sundials} integrator's nonlinear solver convergence test function. See
    section \ref{ss:sunnonlinsol_sunsuppliedfn} for the definition of\\ \noindent
    \id{SUNNonlinSolConvTestFn}.
  \item[ctest\_data] (\id{void*})
    is a data pointer passed to \id{CTestFn} every time it is called.
  \end{args}
}
{
  The return value \id{retval} (of type \id{int}) should be zero for a
  successful call, and a negative value for a failure.
}
{
  {\sunnonlinsol} implementations utilizing their own convergence test
  criteria may set this function to \id{NULL}.
}
% --------------------------------------------------------------------
\ucfunctionf{SUNNonlinSolSetMaxIters}
{
  retval = SUNNonlinSolSetMaxIters(NLS, maxiters);
}
{
  The \textit{optional} function \ID{SUNNonlinSolSetMaxIters} sets the maximum
  number of nonlinear solver iterations. This is typically called by
  {\sundials} integrators to define their default iteration limit, but may be
  adjusted by the user.
}
{
  \begin{args}[maxiters]
  \item[NLS] (\id{SUNNonlinearSolver})
    a {\sunnonlinsol} object.
  \item[maxiters] (\id{int})
    the maximum number of nonlinear iterations.
  \end{args}
}
{
  The return value \id{retval} (of type \id{int}) should be zero for a
  successful call, and a negative value for a failure
  (e.g., $\id{maxiters} < 1$).
}
{}


% ====================================================================
\subsection{SUNNonlinearSolver get functions}
\label{ss:sunnonlinsol_getfn}
% ====================================================================
The following get functions allow {\sundials} integrators to retrieve nonlinear
solver statistics. The routines to get the current total number of
iterations (\id{SUNNonlinSolGetNumIters}) and number of convergence failures
(\id{SUNNonlinSolGetNumConvFails}) are optional. The routine to get
the current nonlinear solver iteration (\id{SUNNonlinSolGetCurIter}) is required
when using the convergence test provided by the {\sundials} integrator
or by the {\arkode} and {\cvode} linear solver interfaces.  Otherwise,
\id{SUNNonlinSolGetCurIter} is optional.

\ucfunctionf{SUNNonlinSolGetNumIters}
{
  retval = SUNNonlinSolGetNumIters(NLS, numiters);
}
{
  The \textit{optional} function \ID{SUNNonlinSolGetNumIters} returns
  the total number of nonlinear solver iterations. This is typically
  called by the {\sundials} integrator to store the nonlinear solver
  statistics, but may also be called by the user.
}
{
  \begin{args}[numiters]
  \item[NLS] (\id{SUNNonlinearSolver})
    a {\sunnonlinsol} object
  \item[numiters] (\id{long int*})
    the total number of nonlinear solver iterations.
  \end{args}
}
{
  The return value \id{retval} (of type \id{int}) should be zero for a
  successful call, and a negative value for a failure.
}
{}
% --------------------------------------------------------------------
\ucfunctionf{SUNNonlinSolGetCurIter}
{
  retval = SUNNonlinSolGetCurIter(NLS, iter);
}
{
  The function \ID{SUNNonlinSolGetCurIter} returns the iteration index
  of the current nonlinear solve. This function is \textit{required}
  when using {\sundials} integrator-provided convergence tests or
  when using a {\sunlinsol} spils linear solver; otherwise it is
  \textit{optional}.
}
{
  \begin{args}[numiters]
  \item[NLS] (\id{SUNNonlinearSolver})
    a {\sunnonlinsol} object
  \item[iter] (\id{int*})
    the nonlinear solver iteration in the current solve starting from
    zero.
  \end{args}
}
{
  The return value \id{retval} (of type \id{int}) should be zero for a
  successful call, and a negative value for a failure.
}
{}
% --------------------------------------------------------------------
\ucfunctionf{SUNNonlinSolGetNumConvFails}
{
  retval = SUNNonlinSolGetNumConvFails(NLS, nconvfails);
}
{
  The \textit{optional} function \ID{SUNNonlinSolGetNumConvFails} returns
  the total number of nonlinear solver convergence failures. This may be
  called by the {\sundials} integrator to store the nonlinear solver
  statistics, but may also be called by the user.
}
{
  \begin{args}[nconvfails]
  \item[NLS] (\id{SUNNonlinearSolver})
    a {\sunnonlinsol} object
  \item[nconvfails] (\id{long int*})
    the total number of nonlinear solver convergence failures.
  \end{args}
}
{
  The return value \id{retval} (of type \id{int}) should be zero for a
  successful call, and a negative value for a failure.
}
{}
% --------------------------------------------------------------------

% ====================================================================
\subsection{Functions provided by SUNDIALS integrators}
\label{ss:sunnonlinsol_sunsuppliedfn}
% ====================================================================

To interface with {\sunnonlinsol} modules, the {\sundials} integrators
supply a variety of routines for evaluating the nonlinear system,
calling the {\sunlinsol} setup and solve functions, and testing the
nonlinear iteration for convergence.  These integrator-provided routines
translate between the user-supplied ODE or DAE systems and the generic
interfaces to the nonlinear or linear systems of equations that result
in their solution. The types for functions provided to a {\sunnonlinsol}
module are defined in the header file
\id{sundials/sundials\_nonlinearsolver.h}, and are described below.
% --------------------------------------------------------------------
\usfunction{SUNNonlinSolSysFn}
{
  typedef int (*SUNNonlinSolSysFn)(N\_Vector ycor, N\_Vector F, void* mem);
}
{
  These functions evaluate the nonlinear system $F(y)$
  for \id{SUNNONLINEARSOLVER\_ROOTFIND} type modules or $G(y)$
  for \id{SUNNONLINEARSOLVER\_FIXEDPOINT} type modules. Memory
  for \id{F} must by be allocated prior to calling this function. The
  vector \id{ycor} will be left unchanged.
}
{
  \begin{args}[ycor]
  \item[ycor]
    is the current correction to the predicted state at which the nonlinear
    system should be evaluated. See section \ref{s:sunnonlinsol_interface}
    for more detail on the nonlinear system formulation.
  \item[F]
    is the output vector containing $F(y)$ or $G(y)$, depending on the
    solver type.
  \item[mem]
    is the {\sundials} integrator memory structure.
  \end{args}
}
{
  The return value \id{retval} (of type \id{int}) is zero for a
  successul solve, a positive value for a recoverable error, and a
  negative value for an unrecoverable error.
  %% The return value \id{retval} (of type \id{int}) will be one of the
  %% following:
  %% \begin{args}[*\_RHSFUNC\_RECVR]
  %% \item[\id{*\_SUCCESS}]
  %%   the function evaluation was successful
  %% \item[\id{*\_RHSFUNC\_RECVR}]
  %%   the ODE right-hand side function returned a recoverable error
  %% \item[\id{*\_RES\_RECVR}]
  %%   the DAE residual function returned a recoverable error
  %% \item[\id{*\_RHSFUNC\_FAIL}]
  %%   the ODE right-hand side function returned an unrecoverable error
  %% \item[\id{*\_RES\_FAIL}]
  %%   the DAE residual function returned an unrecoverable error
  %% \item[\id{*\_MEM\_NULL}]
  %%   the {\sundials} package memory was \id{NULL}
  %% \end{args}
  %% In the above return codes \id{*} is a {\sundials} package-specific
  %% prefix (\id{CV} for {\cvode} or {\cvodes}, \id{IDA} for {\ida} or
  %% {\idas}, and \id{ARK} for {\arkode}).
}
{
  As discussed in section \ref{s:sunnonlinsol_interface}, {\sundials}
  integrators formulate nonlinear systems as a function of the correction to the
  predicted solution. On each call to the nonlinear system function the
  integrator will compute and store the current solution based on the input
  correction. Additionally, the residual will store the value of the ODE
  right-hand side function or DAE residual used in computing the nonlinear
  system residual. These stored values are then directly used in the
  integrator-supplied linear solver setup and solve functions as applicable.
}
% --------------------------------------------------------------------
\usfunction{SUNNonlinSolLSetupFn}
{
  typedef int (*SUNNonlinSolLSetupFn)(&booleantype jbad, booleantype* jcur,\\
                                      &void* mem);
}
{
  These functions are wrappers to the {\sundials} integrator's function
  for setting up linear solves with {\sunlinsol} modules.
}
{
  \begin{args}[jcur]
  \item[jbad]
    is an input indicating whether the nonlinear solver believes that
    $A$ has gone stale (\id{SUNTRUE}) or not (\id{SUNFALSE}).
  \item[jcur]
    is an output indicating whether the routine has updated the
    Jacobian $A$ (\id{SUNTRUE}) or not (\id{SUNFALSE}).
  \item[mem]
    is the {\sundials} integrator memory structure.
  \end{args}
}
{
  The return value \id{retval} (of type \id{int}) is zero for a
  successul solve, a positive value for a recoverable error, and a
  negative value for an unrecoverable error.
}
{
  The \id{SUNNonlinLSetupFn} function sets up the linear system $Ax=b$ where
  $A = \frac{\partial F}{\partial y}$ is the linearization of the nonlinear
  residual function $F(y) = 0$ (when using {\sunlinsol} direct linear solvers)
  or calls the user-defined preconditioner setup function (when using
  {\sunlinsol} iterative linear solvers). {\sunnonlinsol} implementations that
  do not require solving this system, do not utilize {\sunlinsol} linear
  solvers, or use {\sunlinsol} linear solvers that do not require setup may
  ignore these functions.

  As discussed in the description of \id{SUNNonlinSolSysFn}, the linear solver
  setup function assumes that the nonlinear system function has been called
  prior to the linear solver setup function as the setup will utilize saved
  values from the nonlinear system evaluation (e.g., the updated solution).
}
% --------------------------------------------------------------------
\usfunction{SUNNonlinSolLSolveFn}
{
  typedef int (*SUNNonlinSolLSolveFn)(N\_Vector b, void* mem);
}
{
  These functions are wrappers to the {\sundials} integrator's function
  for solving linear systems with {\sunlinsol} modules.
}
{
  \begin{args}[mem]
  \item[b]
    contains the right-hand side vector for the linear solve on input
    and the solution to the linear system on output.
  \item[mem]
    is the {\sundials} integrator memory structure.
  \end{args}
}
{
  The return value \id{retval} (of type \id{int}) is zero for a
  successul solve, a positive value for a recoverable error, and a
  negative value for an unrecoverable error.
}
{
  The \id{SUNNonlinLSolveFn} function solves the linear system $Ax=b$ where
  $A = \frac{\partial F}{\partial y}$ is the linearization of the nonlinear
  residual function $F(y) = 0$. {\sunnonlinsol} implementations that do not
  require solving this system or do not use {\sunlinsol} linear solvers may
  ignore these functions.

  As discussed in the description of \id{SUNNonlinSolSysFn}, the linear solver
  solve function assumes that the nonlinear system function has been called
  prior to the linear solver solve function as the solve may utilize saved
  values from the nonlinear system evaluation (e.g., the updated solution).
}
% --------------------------------------------------------------------
\usfunction{SUNNonlinSolConvTestFn}
{
  typedef int (*SUNNonlinSolConvTestFn)(&SUNNonlinearSolver NLS, N\_Vector ycor,\\
                                        &N\_Vector del, realtype tol,\\
                                        &N\_Vector ewt, void* ctest\_data);
}
{
  These functions are {\sundials} integrator-specific convergence tests for
  nonlinear solvers and are typically supplied by each {\sundials} integrator,
  but users may supply custom problem-specific versions as desired.
}
{
  \begin{args}[ctest\_data]
  \item[NLS]
    is the {\sunnonlinsol} object.
  \item[ycor]
    is the current correction (nonlinear iterate).
  \item[del]
    is the difference between the current and prior nonlinear iterates.
  \item[tol]
    is the nonlinear solver tolerance.
  \item[ewt]
    is the weight vector used in computing weighted norms.
  \item[ctest\_data]
    is the data pointer provided to \id{SUNNonlinSolSetConvTestFn}.
  \end{args}
}
{
  The return value of this routine will be a negative value if an unrecoverable
  error occurred or one of the following:
  \begin{args}[SUN\_NLS\_CONV\_RECVR]
  \item[\id{SUN\_NLS\_SUCCESS}]
    the iteration is converged.
  \item[\id{SUN\_NLS\_CONTINUE}]
    the iteration has not converged, keep iterating.
  \item[\id{SUN\_NLS\_CONV\_RECVR}]
    the iteration appears to be diverging, try to recover.
  \end{args}
}
{
  The tolerance passed to this routine by {\sundials} integrators is the
  tolerance in a weighted root-mean-squared norm with error weight
  vector \id{ewt}. {\sunnonlinsol} modules utilizing their own convergence
  criteria may ignore these functions.
}


% ====================================================================
\subsection{SUNNonlinearSolver return codes}
\label{ss:sunnonlinsol_returncodes}
% ====================================================================

The functions provided to {\sunnonlinsol} modules by each {\sundials}
integrator, and functions within the {\sundials}-provided {\sunnonlinsol}
implementations utilize a common set of return codes, shown below in
Table \ref{t:sunnonlinsol_returncodes}.  Here, negative values
correspond to non-recoverable failures, positive values to recoverable
failures, and zero to a successful call.

\newlength{\ColumnOneA}
\settowidth{\ColumnOneA}{\id{SUN\_NLS\_VECTOROP\_ERR}}
\newlength{\ColumnTwoA}
\settowidth{\ColumnTwoA}{\id{Value}}
\newlength{\ColumnThreeA}
\setlength{\ColumnThreeA}{\textwidth}
\addtolength{\ColumnThreeA}{-0.5in}
\addtolength{\ColumnThreeA}{-\ColumnOneA}
\addtolength{\ColumnThreeA}{-\ColumnTwoA}

\tablecaption{Description of the \id{SUNNonlinearSolver} return codes}\label{t:sunnonlinsol_returncodes}
\tablefirsthead{\hline {\rule{0mm}{5mm}}{\bf Name} & {\bf Value} & {\bf Description} \\[3mm] \hline\hline}
\tablehead{\hline \multicolumn{3}{|l|}{\small\slshape continued from last page} \\
           \hline {\rule{0mm}{5mm}}{\bf Name} & {\bf Value} & {\bf Description} \\[3mm] \hline\hline}
%% \tablehead{\hline {\rule{0mm}{5mm}}{\bf Name} & {\bf Value} & {\bf Description} \\[3mm] \hline\hline}
\tabletail{\hline \multicolumn{3}{|r|}{\small\slshape continued on next page} \\ \hline}
\begin{xtabular}{|p{\ColumnOneA}|p{\ColumnTwoA}|p{\ColumnThreeA}|}
%%
\id{SUN\_NLS\_SUCCESS}        & \id{0}  & successful call or converged solve
\\[1mm]
%%
\id{SUN\_NLS\_CONTINUE}       & \id{901}  & the nonlinear solver is not
                                            converged, keep iterating
\\[1mm]
%%
\id{SUN\_NLS\_CONV\_RECVR}    & \id{902}  & the nonlinear solver appears to
                                            be diverging, try to recover
\\[1mm]
%%
\id{SUN\_NLS\_MEM\_NULL}      & \id{-901} & a memory argument is \id{NULL}
\\[1mm]
%%
\id{SUN\_NLS\_MEM\_FAIL}      & \id{-902} & a memory access or allocation failed
\\[1mm]
%%
\id{SUN\_NLS\_ILL\_INPUT}     & \id{-903} & an illegal input option was provided
\\
\id{SUN\_NLS\_VECTOROP\_ERR}  & \id{-904} & a {\nvector} operation failed
\\
\id{SUN\_NLS\_EXT\_FAIL}      & \id{-905} & an external library call returned an error
\\
\hline
\end{xtabular}
\bigskip


% ====================================================================
\subsection{The generic SUNNonlinearSolver module}
\label{ss:sunnonlinsol_generic}
% ====================================================================

{\sundials} integrators interact with specific {\sunnonlinsol}
implementations through the generic {\sunnonlinsol} module on which all
other {\sunnonlinsol} implementations are built. The \\ \noindent
\id{SUNNonlinearSolver} type is a pointer to a structure containing an
implementation-dependent \textit{content} field and an \textit{ops}
field. The type \id{SUNNonlinearSolver} is defined as follows:
%%
%%
\begin{verbatim}
typedef struct _generic_SUNNonlinearSolver *SUNNonlinearSolver;

struct _generic_SUNNonlinearSolver {
  void *content;
  struct _generic_SUNNonlinearSolver_Ops *ops;
};
\end{verbatim}
%%
%%
where the \id{\_generic\_SUNNonlinearSolver\_Ops} structure is a list of
pointers to the various actual nonlinear solver operations provided by a
specific implementation. The \id{\_generic\_SUNNonlinearSolver\_Ops}
structure is defined as
%%
%%
\begin{verbatim}
struct _generic_SUNNonlinearSolver_Ops {
  SUNNonlinearSolver_Type (*gettype)(SUNNonlinearSolver);
  int                     (*initialize)(SUNNonlinearSolver);
  int                     (*setup)(SUNNonlinearSolver, N_Vector, void*);
  int                     (*solve)(SUNNonlinearSolver, N_Vector, N_Vector,
                                   N_Vector, realtype, booleantype, void*);
  int                     (*free)(SUNNonlinearSolver);
  int                     (*setsysfn)(SUNNonlinearSolver, SUNNonlinSolSysFn);
  int                     (*setlsetupfn)(SUNNonlinearSolver, SUNNonlinSolLSetupFn);
  int                     (*setlsolvefn)(SUNNonlinearSolver, SUNNonlinSolLSolveFn);
  int                     (*setctestfn)(SUNNonlinearSolver, SUNNonlinSolConvTestFn,
                                        void*);
  int                     (*setmaxiters)(SUNNonlinearSolver, int);
  int                     (*getnumiters)(SUNNonlinearSolver, long int*);
  int                     (*getcuriter)(SUNNonlinearSolver, int*);
  int                     (*getnumconvfails)(SUNNonlinearSolver, long int*);
};
\end{verbatim}
%%
%%
The generic {\sunnonlinsol} module defines and implements the nonlinear
solver operations defined in Sections \ref{ss:sunnonlinsol_corefn}
-- \ref{ss:sunnonlinsol_getfn}. These routines are in fact only
wrappers to the nonlinear solver operations provided by a particular
{\sunnonlinsol} implementation, which are accessed through the ops
field of the \id{SUNNonlinearSolver} structure. To illustrate this
point we show below the implementation of a typical nonlinear solver
operation from the generic {\sunnonlinsol} module, namely
\id{SUNNonlinSolSolve}, which solves the nonlinear system and returns a flag
denoting a successful or failed solve:
%%
%%
\begin{verbatim}
int SUNNonlinSolSolve(SUNNonlinearSolver NLS,
                      N_Vector y0, N_Vector y,
                      N_Vector w, realtype tol,
                      booleantype callLSetup, void* mem)
{
  return((int) NLS->ops->solve(NLS, y0, y, w, tol, callLSetup, mem));
}
\end{verbatim}

The Fortran 2003 interface provides a \id{bind(C)} derived-type for the
\id{\_generic\_SUNNonlinearSolver} and the \id{\_generic\_SUNNonlinearSolver\_Ops} structures.
Their definition is given below.
%%
%%
\begin{verbatim}
 type, bind(C), public :: SUNNonlinearSolver
  type(C_PTR), public :: content
  type(C_PTR), public :: ops
 end type SUNNonlinearSolver

 type, bind(C), public :: SUNNonlinearSolver_Ops
  type(C_FUNPTR), public :: gettype
  type(C_FUNPTR), public :: initialize
  type(C_FUNPTR), public :: setup
  type(C_FUNPTR), public :: solve
  type(C_FUNPTR), public :: free
  type(C_FUNPTR), public :: setsysfn
  type(C_FUNPTR), public :: setlsetupfn
  type(C_FUNPTR), public :: setlsolvefn
  type(C_FUNPTR), public :: setctestfn
  type(C_FUNPTR), public :: setmaxiters
  type(C_FUNPTR), public :: getnumiters
  type(C_FUNPTR), public :: getcuriter
  type(C_FUNPTR), public :: getnumconvfails
 end type SUNNonlinearSolver_Ops
\end{verbatim}


% ====================================================================
\subsection{Usage with sensitivity enabled integrators}
\label{ss:sunnonlinsol_sens}
% ====================================================================

When used with {\sundials} packages that support sensitivity analysis
capabilities (e.g., {\cvodes} and {\idas}) a special {\nvector} module is used
to interface with {\sunnonlinsol} modules for solves involving sensitivity
vectors stored in an {\nvector} array. As described below, the {\nvecwrap}
module is an {\nvector} implementation where the vector content is an {\nvector}
array. This wrapper vector allows {\sunnonlinsol} modules to operate on data
stored as a collection of vectors.

For all {\sundials}-provided {\sunnonlinsol} modules a special constructor
wrapper is provided so users do not need to interact directly with the
{\nvecwrap} module. These constructors follow the naming
convention \id{SUNNonlinSol\_***Sens(count,...)} where \id{***} is the name of
the {\sunnonlinsol} module, \id{count} is the size of the vector wrapper,
and \id{...} are the module-specific constructor arguments.

% ====================================================================
\subsubsection*{The NVECTOR\_SENSWRAPPER module}
\label{ss:sunnonlinsol_senswrapper}
% ====================================================================

This section describes the {\nvecwrap} implementation of an {\nvector}. To
access the {\nvecwrap} module, include the header file \\ \noindent
\id{sundials/sundials\_nvector\_senswrapper.h}.

The {\nvecwrap} module defines an \id{N\_Vector} implementing all of the
standard vectors operations defined in Table \ref{ss:nvecops} but with some
changes to how operations are computed in order to accommodate operating on a
collection of vectors.
\begin{enumerate}

\item Element-wise vector operations are computed on a vector-by-vector basis. For
example, the linear sum of two wrappers containing $n_v$ vectors of length $n$,
\id{N\_VLinearSum(a,x,b,y,z)}, is computed as
\begin{equation*}
z_{j,i} = a x_{j,i} + b y_{j,i}, \quad i=0,\ldots,n-1, \quad j=0,\ldots,n_v-1.
\end{equation*}

\item The dot product of two wrappers containing $n_v$ vectors of length $n$ is
computed as if it were the dot product of two vectors of length $n n_v$.
Thus \id{d = N\_VDotProd(x,y)} is
\begin{equation*}
d = \sum_{j=0}^{n_v-1} \sum_{i=0}^{n-1} x_{j,i} y_{j,i}.
\end{equation*}

\item All norms are computed as the maximum of the individual norms of the $n_v$ vectors
in the wrapper. For example, the weighted root mean square norm
\id{m = N\_VWrmsNorm(x, w)} is
\begin{equation*}
m = \max_{j} \sqrt{ \left( \frac1n \sum_{i=0}^{n-1} \left(x_{j,i}
w_{j,i}\right)^2\right) }
\end{equation*}

\end{enumerate}
To enable usage alongside other {\nvector} modules the {\nvecwrap} functions
implementing vector operations have \id{\_SensWrapper} appended to the
generic vector operation name.

The {\nvecwrap} module provides the following constructors for creating an
{\nvecwrap}:
\ucfunctionf{N\_VNewEmpty\_SensWrapper}
{
  w = N\_VNewEmpty\_SensWrapper(count);
}
{
  The function \ID{N\_VNewEmpty\_SensWrapper} creates an empty {\nvecwrap}
  wrapper with space for \id{count} vectors.
}
{
  \begin{args}[count]
  \item[count] (\id{int})
    the number of vectors the wrapper will contain.
  \end{args}
}
{
  The return value \id{w} (of type \id{N\_Vector}) will be a {\nvector} object
  if the constructor exits successfully, otherwise \id{w} will be \id{NULL}.
}
{}
%
\ucfunctionf{N\_VNew\_SensWrapper}
{
  w = N\_VNew\_SensWrapper(count, y);
}
{
  The function \ID{N\_VNew\_SensWrapper} creates an {\nvecwrap}
  wrapper containing \id{count} vectors cloned from \id{y}.
}
{
  \begin{args}[count]
  \item[count] (\id{int})
    the number of vectors the wrapper will contain.
  \item[y] (\id{N\_Vector})
    the template vectors to use in creating the vector wrapper.
  \end{args}
}
{
  The return value \id{w} (of type \id{N\_Vector}) will be a {\nvector} object
  if the constructor exits successfully, otherwise \id{w} will be \id{NULL}.
}
{}

The {\nvecwrap} implementation of the {\nvector} module defines
the \textit{content} field of the \id{N\_Vector} to be a structure containing
an \id{N\_Vector} array, the number of vectors in the vector array, and a
boolean flag indicating ownership of the vectors in the vector array.
%
\begin{verbatim}
struct _N_VectorContent_SensWrapper {
  N_Vector* vecs;
  int nvecs;
  booleantype own_vecs;
};
\end{verbatim}
%
The following macros are provided to access the content of an {\nvecwrap}
vector.
\begin{itemize}
 \item \id{NV\_CONTENT\_SW(v)}   - provides access to the content structure
 \item \id{NV\_VECS\_SW(v)}      - provides access to the vector array
 \item \id{NV\_NVECS\_SW(v)}     - provides access to the number of vectors
 \item \id{NV\_OWN\_VECS\_SW(v)} - provides access to the ownership flag
 \item \id{NV\_VEC\_SW(v,i)}     - provides access to the \id{i}-th vector in
 the  vector array
\end{itemize}

% ====================================================================
\subsection{Implementing a Custom SUNNonlinearSolver Module}
\label{ss:sunnonlinsol_custom}
% ====================================================================

A {\sunnonlinsol} implementation \textit{must} do the following:
\begin{enumerate}
\item Specify the content of the {\sunnonlinsol} module.
\item Define and implement the required nonlinear solver operations
  defined in Sections \ref{ss:sunnonlinsol_corefn}
  -- \ref{ss:sunnonlinsol_getfn}. Note that the names of the module
  routines should be unique to that implementation in order to permit
  using more than one {\sunnonlinsol} module (each with different
  \id{SUNNonlinearSolver} internal data representations) in
  the same code.
\item Define and implement a user-callable constructor to create a
  \id{SUNNonlinearSolver} object.
\end{enumerate}
Additionally, a \id{SUNNonlinearSolver} implementation \textit{may} do
the following:
\begin{enumerate}
\item Define and implement additional user-callable ``set''
  routines acting on the \id{SUNNonlinearSolver} object, e.g., for
  setting various configuration options to tune the performance of
  the nonlinear solve algorithm.
\item Provide additional user-callable ``get'' routines acting on the
  \id{SUNNonlinearSolver} object, e.g., for returning various solve
  statistics.
\end{enumerate}

To aid in the creation of custom {\sunnonlinsol} modules the generic
{\sunnonlinsol} module provides the utility functions \id{SUNNonlinSolNewEmpty}
and \id{SUNNonlinsolFreeEmpty}. When used in custom {\sunnonlinsol} constructors,
the function \id{SUNNonlinSolNewEmpty} will ease the introduction of any new
optional nonlinear solver operations to the {\sunnonlinsol} API by ensuring only
required operations need to be set.
%
% --------------------------------------------------------------------
%
\ucfunctionf{SUNNonlinSolNewEmpty}
{
  NLS = SUNNonlinSolNewEmpty();
}
{
  The function \Id{SUNNonlinSolNewEmpty} allocates a new generic {\sunnonlinsol}
  object and initializes its content pointer and the function pointers in the
  operations structure to \id{NULL}.
}
{}
{
  This function returns a \id{SUNNonlinearSolver} object. If an error occurs when
  allocating the object, then this routine will return \id{NULL}.
}
{}
%
% --------------------------------------------------------------------
%
\ucfunctionf{SUNNonlinSolFreeEmpty}
{
  SUNNonlinSolFreeEmpty(NLS);
}
{
  This routine frees the generic \id{SUNNonlinearSolver} object, under the assumption that any
  implementation-specific data that was allocated within the underlying content structure
  has already been freed. It will additionally test whether the ops pointer is \id{NULL},
  and, if it is not, it will free it as well.
}
{
  \begin{args}[NLS]
  \item[NLS] (\id{SUNNonlinearSolver})
  \end{args}
}
{}
{}


%---------------------------------------------------------------------------
\section{IDAS SUNNonlinearSolver interface}
\label{s:sunnonlinsol_interface}
%---------------------------------------------------------------------------

As discussed in Chapter \ref{s:math} each integration step requires the
(approximate) solution of the nonlinear system
\begin{equation}
  G(y_n) = F\left(t_n, y_n, h_{n}^{-1}\sum_{i=0}^{q}\alpha_{n,i}y_{n-i}\right) = 0.
\end{equation}
Rather than solving this system for the new state $y_n$ {\idas} reformulates the
system to solve for the correction $y_{cor}$ to the predicted new state $y_{pred}$
and its derivative $\dot{y}_{pred}$ so that $y_n = y_{pred} + y_{cor}$ and
$\dot{y}_n = \dot{y}_{pred} + h_{n}^{-1} \alpha_{n,0} y_{cor}$. The nonlinear
system rewritten in terms of $y_{cor}$ is
\begin{equation} \label{eq:res_corrector}
  G(y_{cor}) = F\left(t_n, y_{pred}+y_{cor},
  \dot{y}_{pred} + \alpha y_{cor}\right) = 0.
\end{equation}
where $\alpha = h_{n}^{-1} \alpha_{n,0}$.
Similarly in the forward sensitivity analysis case the nonlinear system is also
reformulated in terms of the correction to the predicted sensitivities.

The nonlinear system function provided by {\idas} to the nonlinear solver module
internally updates the current value of the new state and its derivative based
on the current corretion passed to the function (as well as the sensitivities).
These values are used when calling the DAE residual function and when setting up
linear solves (e.g., for updating the Jacobian or preconditioner).

{\idas} provides several advanced functions that will not be needed by most users,
but might be useful for users who choose to provide their own implementation of
the \id{SUNNonlinearSolver} API. For example, such a user might need access to
the current $y$ and $\dot{y}$ vectors to compute Jacobian data.

\ucfunctionf{IDAGetCurrentCj}
{
  flag = IDAGetCurrentCj(ida\_mem, \&cj);
}
{
  The function \ID{IDAGetCurrentCj} returns the scalar $c_j$
  which is proportional to the inverse of the step size ($\alpha$ in
  Eq. \ref{e:DAE_Jacobian}).
}
{
  \begin{args}[ida\_mem]
  \item[ida\_mem] (\id{void *})
    pointer to the {\idas} memory block.
  \item[cj] (\id{realtype})
    the value of $c_j$.
  \end{args}
}
{
  The return value \id{flag} (of type \id{int}) is one of
  \begin{args}[IDA\_MEM\_NULL]
  \item[IDA\_SUCCESS]
    The optional output value has been successfully set.
  \item[IDA\_MEM\_NULL]
    The \id{ida\_mem} pointer is \id{NULL}.
  \end{args}
}
{}

\ucfunctionf{IDAGetCurrentY}
{
  flag = IDAGetCurrentY(ida\_mem, \&y);
}
{
  The function \ID{IDAGetCurrentY} returns the current $y$ vector.
}
{
  \begin{args}[ida\_mem]
  \item[ida\_mem] (\id{void *})
    pointer to the {\idas} memory block.
  \item[y] (\id{N\_Vector *})
    the current $y$ vector
  \end{args}
}
{
  The return value \id{flag} (of type \id{int}) is one of
  \begin{args}[IDA\_MEM\_NULL]
  \item[IDA\_SUCCESS]
    The optional output value has been successfully set.
  \item[IDA\_MEM\_NULL]
    The \id{ida\_mem} pointer is \id{NULL}.
  \end{args}
}
{}

\ucfunctionf{IDAGetCurrentYp}
{
  flag = IDAGetCurrentYp(ida\_mem, \&yp);
}
{
  The function \ID{IDAGetCurrentYp} returns the current $\dot{y}$ vector.
}
{
  \begin{args}[ida\_mem]
  \item[ida\_mem] (\id{void *})
    pointer to the {\idas} memory block.
  \item[yp] (\id{N\_Vector *})
    the current $\dot{y}$ vector
  \end{args}
}
{
  The return value \id{flag} (of type \id{int}) is one of
  \begin{args}[IDA\_MEM\_NULL]
  \item[IDA\_SUCCESS]
    The optional output value has been successfully set.
  \item[IDA\_MEM\_NULL]
    The \id{ida\_mem} pointer is \id{NULL}.
  \end{args}
}
{}

\ucfunctionf{IDAGetCurrentYSens}
{
  flag = IDAGetCurrentYSens(ida\_mem, \&yyS);
}
{
  The function \ID{IDAGetCurrentYSens} returns the current sensitivity
  vector array.
}
{
  \begin{args}[ida\_mem]
  \item[ida\_mem] (\id{void *})
    pointer to the {\idas} memory block.
  \item[yyS] (\id{N\_Vector **})
    pointer to the vector array that is set to the array of sensitivity vectors
  \end{args}
}
{
  The return value \id{flag} (of type \id{int}) is one of
  \begin{args}[IDA\_MEM\_NULL]
  \item[IDA\_SUCCESS]
    The optional output value has been successfully set.
  \item[IDA\_MEM\_NULL]
    The \id{ida\_mem} pointer is \id{NULL}.
  \end{args}
}
{}

\ucfunctionf{IDAGetCurrentYpSens}
{
  flag = IDAGetCurrentYpSens(ida\_mem, \&ypS);
}
{
  The function \ID{IDAGetCurrentYpSens} returns the derivative the current
  sensitivity vector array.
}
{
  \begin{args}[ida\_mem]
  \item[ida\_mem] (\id{void *})
    pointer to the {\idas} memory block.
  \item[ypS] (\id{N\_Vector **})
    pointer to the vector array that is set to the array of sensitivity
    vector derivatives
  \end{args}
}
{
  The return value \id{flag} (of type \id{int}) is one of
  \begin{args}[IDA\_MEM\_NULL]
  \item[IDA\_SUCCESS]
    The optional output value has been successfully set.
  \item[IDA\_MEM\_NULL]
    The \id{ida\_mem} pointer is \id{NULL}.
  \end{args}
}
{}

\ucfunctionf{IDAGetNonlinearSystemData}
{
  flag = IDAGetNonlinearSystemData(&ida\_mem, \&tcur, \&yypred, \&yppred, \\
                                   &\&yyn, \&ypn, \&res, \&cj, user\_data);
}
{
  The function \ID{IDAGetNonlinearSystemData} returns all internal
  data required to construct the current nonlinear system
  \eqref{eq:res_corrector}.
}
{
  \begin{args}[user\_data]
  \item[ida\_mem] (\id{void *}) pointer to the {\idas} memory block.
  \item[tcur] (\id{realtype*}) current value of the independent variable $t_n$.
  \item[yypred] (\id{N\_Vector*}) predicted value of $y_{pred}$ at $t_n$.
  \item[yppred] (\id{N\_Vector*}) predicted value of $\dot{y}_{pred}$ at $t_n$.
  \item[yyn] (\id{N\_Vector*}) the vector $y_n$. This vector may be not current
    and may need to be filled (see the note below).
  \item[ypn] (\id{N\_Vector*}) the vector $\dot{y}_n$. This vector may be not
    current and may need to be filled (see the note below).
  \item[res] (\id{N\_Vector*}) the resiudal function evaluated at the current
    time and state, $F(t_n, y_n, \dot{y}_n)$. This vector may be not current and
    may need to be filled (see the note below).
  \item[cj] (\id{realtype*}) the scalar $c_j$ which is proportional to the
  inverse of the step size ($\alpha$ in \eqref{eq:res_corrector}).
  \item[user\_data] (\id{void**}) pointer to the user-defined data structures
  \end{args}
}
{
  The return value \id{flag} (of type \id{int}) is one of
  \begin{args}[IDA\_MEM\_NULL]
  \item[IDA\_SUCCESS]
    The optional output values have been successfully set.
  \item[IDA\_MEM\_NULL]
    The \id{ida\_mem} pointer is \id{NULL}.
  \end{args}
}
{
  This routine is intended for users who wish to attach a custom
  \id{SUNNonlinSolSysFn} (see \S\ref{ss:sunnonlinsol_sunsuppliedfn}) to an
  existing \id{SUNNonlinearSolver} object (through a call to
  \id{SUNNonlinSolSetSysFn}) or who need access to nonlinear system data to
  compute the nonlinear system fucntion as part of a custom
  \id{SUNNonlinearSolver} object.

  When supplying a custom \id{SUNNonlinSolSysFn} to an existing
  \id{SUNNonlinearSolver} object, the user should call
  \id{IDAGetNonlinearSystemData} \textbf{inside} the nonlinear system
  function to access the requisite data for evaluting the nonlinear system
  function of their choosing. Additionlly, if the \id{SUNNonlinearSolver} object
  (existing or custom) leverages the \id{SUNNonlinSolLSetupFn} and/or
  \id{SUNNonlinSolLSolveFn} functions supplied by {\idas} (through calls to
  \id{SUNNonlinSolSetLSetupFn} and \id{SUNNonlinSolSetLSolveFn} respectively)
  the vectors \id{yyn}, \id{ypn}, and \id{res} \textbf{must be filled} in by the
  user's \id{SUNNonlinSolSysFn} with the current state and corresponding
  evaluation of the right-hand side function respectively i.e.,
  \begin{align*}
    yyn &= y_{pred} + y_{cor}, \\
    ypn &= \dot{y}_{pred} + \alpha \dot{y}_{cor}, \\
    res &= F\left(t_{n}, y_n, \dot{y}_n\right),
  \end{align*}
  where $y_{cor}$ was the first argument supplied to the \id{SUNNonlinSolSysFn}.

  If this function is called as part of a custom linear solver (i.e., the
  default \id{SUNNonlinSolSysFn} is used) then the vectors \id{yyn}, \id{ypn},
  and \id{res} are only current when \id{IDAGetNonlinearSystemData} is called
  after an evaluation of the nonlinear system function.
}

\ucfunctionf{IDAGetNonlinearSystemDataSens}
{
  flag = IDAGetNonlinearSystemDataSens(&ida\_mem, \&tcur, \&yySpred, \&ypSpred,\\
                                       &\&yySn, \&ypSn, \&cj, user\_data);
}
{
  The function \ID{IDAGetNonlinearSystemDataSens} returns all internal
  sensitivity data required to construct the current nonlinear system
  \eqref{eq:res_corrector}.
}
{
  \begin{args}[user\_data]
  \item[ida\_mem] (\id{void *}) pointer to the {\idas} memory block.
  \item[tcur] (\id{realtype*}) current value of the independent variable $t_n$.
  \item[yySpred] (\id{N\_Vector**}) predicted value of $yS_{i,pred}$ at $t_n$
    for $i = 0 \dots N_s - 1$.
  \item[ypSpred] (\id{N\_Vector**}) predicted value of $\dot{y}S_{i,pred}$ at
    $t_n$ for $i = 0 \dots N_s - 1$.
  \item[yySn] (\id{N\_Vector**}) the vectors $yS_{i,n}$. These vectors may be
    not current (see the note below).
  \item[ypSn] (\id{N\_Vector**}) the vectors $\dot{y}S_{i,n}$. These vectors may
    be not current (see the note below).
  \item[cj] (\id{realtype*}) the scalar $c_j$ which is proportional to the
  inverse of the step size ($\alpha$ in \eqref{eq:res_corrector}).
  \item[user\_data] (\id{void**}) pointer to the user-defined data structures
  \end{args}
}
{
  The return value \id{flag} (of type \id{int}) is one of
  \begin{args}[IDA\_MEM\_NULL]
  \item[IDA\_SUCCESS]
    The optional output values have been successfully set.
  \item[IDA\_MEM\_NULL]
    The \id{ida\_mem} pointer is \id{NULL}.
  \end{args}
}
{
  This routine is intended for users who wish to attach a custom
  \id{SUNNonlinSolSysFn} (see \S\ref{ss:sunnonlinsol_sunsuppliedfn}) to an
  existing \id{SUNNonlinearSolver} object (through a call to
  \id{SUNNonlinSolSetSysFn}) or who need access to nonlinear system data to
  compute the nonlinear system fucntion as part of a custom
  \id{SUNNonlinearSolver} object.

  When supplying a custom \id{SUNNonlinSolSysFn} to an existing
  \id{SUNNonlinearSolver} object, the user should call
  \id{IDAGetNonlinearSystemDataSens} \textbf{inside} the nonlinear system
  function to access the requisite data for evaluting the nonlinear system
  function of their choosing. Additionlly, if the the vectors \id{yySn} and
  \id{ypSn} are provided as additional workspace and do not need to be filled in
  by the user's \id{SUNNonlinSolSysFn}.

  If this function is called as part of a custom linear solver (i.e., the
  default \id{SUNNonlinSolSysFn} is used) then the vectors \id{yySn} and
  \id{ypSn} are only current when \id{IDAGetNonlinearSystemDataSens} is
  called after an evaluation of the nonlinear system function.
}

\ucfunctionf{IDAComputeY}
{
  flag = IDAComputeY(ida\_mem, ycor, y);
}
{
  The function computes the current $y(t)$ vector based on the given correction
  vector from the nonlinear solver.
}
{
  \begin{args}[ida\_mem]
    \item[ida\_mem] (\id{void *}) pointer to the {\idas} memory block
    \item[ycor] (\id{N\_Vector}) the correction
    \item[y] (\id{N\_Vector}) the output vector
  \end{args}
}
{
  The return value \id{flag} (of type \id{int}) is one of
  \begin{args}[IDA\_MEM\_NULL]
  \item[IDA\_SUCCESS]
    The optional output value has been successfully set.
  \item[\Id{IDA\_MEM\_NULL}]
    The \id{ida\_mem} pointer is \id{NULL}.
  \end{args}
}
{}

\ucfunctionf{IDAComputeYp}
{
  flag = IDAComputeYp(ida\_mem, ycor, yp);
}
{
  The function computes $\dot{y}(t)$ based on the given correction
  vector from the nonlinear solver.
}
{
  \begin{args}[ida\_mem]
    \item[ida\_mem] (\id{void *}) pointer to the {\idas} memory block
    \item[ycor] (\id{N\_Vector}) the correction
    \item[yp] (\id{N\_Vector}) the output vector array
  \end{args}
}
{
  The return value \id{flag} (of type \id{int}) is one of
  \begin{args}[IDA\_MEM\_NULL]
  \item[IDA\_SUCCESS]
    The optional output value has been successfully set.
  \item[IDA\_MEM\_NULL]
    The \id{ida\_mem} pointer is \id{NULL}.
  \end{args}
}
{}

\ucfunctionf{IDAComputeYSens}
{
  flag = IDAComputeYSens(ida\_mem, ycorS, yys);
}
{
  The function computes the sensitivities based on the given correction
  vector from the nonlinear solver.
}
{
  \begin{args}[ida\_mem]
    \item[ida\_mem] (\id{void *}) pointer to the {\idas} memory block
    \item[ycorS] (\id{N\_Vector} *) the correction
    \item[yyS] (\id{N\_Vector} *) the output vector array
  \end{args}
}
{
  The return value \id{flag} (of type \id{int}) is one of
  \begin{args}[IDA\_MEM\_NULL]
  \item[IDA\_SUCCESS]
    The optional output value has been successfully set.
  \item[IDA\_MEM\_NULL]
    The \id{ida\_mem} pointer is \id{NULL}.
  \end{args}
}
{}

\ucfunctionf{IDAComputeYpSens}
{
  flag = IDAComputeYpSens(ida\_mem, ycorS, ypS);
}
{
  The function computes the sensitivity derivatives based on the
  given correction vector from the nonlinear solver.
}
{
  \begin{args}[ida\_mem]
    \item[ida\_mem] (\id{void *}) pointer to the {\idas} memory block
    \item[ycorS] (\id{N\_Vector *}) the correction
    \item[ypS] (\id{N\_Vector *}) the output vector array
  \end{args}
}
{
  The return value \id{flag} (of type \id{int}) is one of
  \begin{args}[IDA\_MEM\_NULL]
  \item[IDA\_SUCCESS]
    The optional output value has been successfully set.
  \item[IDA\_MEM\_NULL]
    The \id{ida\_mem} pointer is \id{NULL}.
  \end{args}
}
{}

%---------------------------------------------------------------------------
% sunnonlinsol module sections
%---------------------------------------------------------------------------

% ====================================================================
\section{The SUNNonlinearSolver\_Newton implementation}
\label{s:sunnonlinsol_newton}
% ====================================================================

This section describes the {\sunnonlinsol} implementation of Newton's method. To
access the {\sunnonlinsolnewton} module, include the header file
\id{sunnonlinsol/sunnonlinsol\_newton.h}. We note that the {\sunnonlinsolnewton}
module is accessible from {\sundials} integrators \textit{without} separately
linking to the \id{libsundials\_sunnonlinsolnewton} module library.

% ====================================================================
\subsection{SUNNonlinearSolver\_Newton description}
\label{ss:sunnonlinsolnewton_math}
% ====================================================================

To find the solution to
\begin{equation}\label{e:newton_sys}
  F(y) = 0 \,
\end{equation}
given an initial guess $y^{(0)}$, Newton's method computes a series of
approximate solutions
\begin{equation}
  y^{(m+1)} = y^{(m)} + \delta^{(m+1)}
\end{equation}
where $m$ is the Newton iteration index, and the Newton update $\delta^{(m+1)}$
is the solution of the linear system
\begin{equation}\label{e:newton_linsys}
  A(y^{(m)}) \delta^{(m+1)} = -F(y^{(m)}) \, ,
\end{equation}
in which $A$ is the Jacobian matrix
\begin{equation}\label{e:newton_mat}
  A \equiv \partial F / \partial y \, .
\end{equation}
Depending on the linear solver used, the {\sunnonlinsolnewton} module
will employ either a Modified Newton method, or an Inexact Newton
method~\cite{Bro:87,BrSa:90,DES:82,DeSc:96,Kel:95}. When used with a direct
linear solver, the Jacobian matrix $A$ is held constant during the Newton
iteration, resulting in a Modified Newton method. With a matrix-free iterative
linear solver, the iteration is an Inexact Newton method.

In both cases, calls to the integrator-supplied \id{SUNNonlinSolLSetupFn}
function are made infrequently to amortize the increased cost of
matrix operations (updating $A$ and its factorization within direct
linear solvers, or updating the preconditioner within iterative linear
solvers).  Specifically, {\sunnonlinsolnewton} will call the
\id{SUNNonlinSolLSetupFn} function in two instances:
\begin{itemize}
\item[(a)] when requested by the integrator (the input
  \id{callLSetSetup} is \id{SUNTRUE}) before attempting the Newton
  iteration, or
\item[(b)] when reattempting the nonlinear solve after a recoverable
  failure occurs in the Newton iteration with stale Jacobian
  information (\id{jcur} is \id{SUNFALSE}).  In this case,
  {\sunnonlinsolnewton} will set \id{jbad} to \id{SUNTRUE} before
  calling the \id{SUNNonlinSolLSetupFn} function.
\end{itemize}
Whether the Jacobian matrix $A$ is fully or partially updated depends
on logic unique to each integrator-supplied \id{SUNNonlinSolSetupFn}
routine. We refer to the discussion of nonlinear solver strategies
provided in Chapter \ref{s:math} for details on this decision.

The default maximum number of iterations and the stopping criteria for
the Newton iteration are supplied by the {\sundials} integrator when
{\sunnonlinsolnewton} is attached to it.  Both the maximum number of
iterations and the convergence test function may be modified by the
user by calling the \id{SUNNonlinSolSetMaxIters} and/or
\id{SUNNonlinSolSetConvTestFn} functions after attaching the
{\sunnonlinsolnewton} object to the integrator.

% ====================================================================
\subsection{SUNNonlinearSolver\_Newton functions}
\label{ss:sunnonlinsolnewton_functions}
% ====================================================================

The {\sunnonlinsolnewton} module provides the following constructors
for creating a \newline
\id{SUNNonlinearSolver} object.
% --------------------------------------------------------------------
\ucfunctionf{SUNNonlinSol\_Newton}
{
  NLS = SUNNonlinSol\_Newton(y);
}
{
  The function \ID{SUNNonlinSol\_Newton} creates a
  \id{SUNNonlinearSolver} object for use with {\sundials} integrators to
  solve nonlinear systems of the form $F(y) = 0$ using Newton's method.
}
{
  \begin{args}[y]
  \item[y] (\id{N\_Vector})
    a template for cloning vectors needed within the solver.
  \end{args}
}
{
  The return value \id{NLS} (of type \id{SUNNonlinearSolver}) will be
  a {\sunnonlinsol} object if the constructor exits successfully,
  otherwise \id{NLS} will be \id{NULL}.
}
{}
% --------------------------------------------------------------------
\ucfunctionf{SUNNonlinSol\_NewtonSens}
{
  NLS = SUNNonlinSol\_NewtonSens(count, y);
}
{
  The function \ID{SUNNonlinSol\_NewtonSens} creates a
  \id{SUNNonlinearSolver} object for use with {\sundials} sensitivity enabled
  integrators ({\cvodes} and {\idas}) to solve nonlinear systems of the form
  $F(y) = 0$ using Newton's method.
}
{
  \begin{args}[count]
  \item[count] (\id{int})
    the number of vectors in the nonlinear solve. When integrating a system
    containing \id{Ns} sensitivities the value of \id{count} is:
    \begin{itemize}
      \item \id{Ns+1} if using a \textit{simultaneous} corrector approach.
      \item \id{Ns} if using a \textit{staggered} corrector approach.
    \end{itemize}
  \item[y] (\id{N\_Vector})
    a template for cloning vectors needed within the solver.
  \end{args}
}
{
  The return value \id{NLS} (of type \id{SUNNonlinearSolver}) will be
  a {\sunnonlinsol} object if the constructor exits successfully,
  otherwise \id{NLS} will be \id{NULL}.
}
{}
% --------------------------------------------------------------------
The {\sunnonlinsolnewton} module implements all of the functions
defined in sections \ref{ss:sunnonlinsol_corefn} --
\ref{ss:sunnonlinsol_getfn} except for the \id{SUNNonlinSolSetup} function. The
{\sunnonlinsolnewton} functions have the same names as those defined
by the generic {\sunnonlinsol} API with \id{\_Newton} appended to the
function name. Unless using the {\sunnonlinsolnewton} module as a
standalone nonlinear solver the generic functions defined in sections
\ref{ss:sunnonlinsol_corefn} -- \ref{ss:sunnonlinsol_getfn} should be
called in favor of the {\sunnonlinsolnewton}-specific implementations.

The {\sunnonlinsolnewton} module also defines the following additional
user-callable function.
% --------------------------------------------------------------------
\ucfunctionf{SUNNonlinSolGetSysFn\_Newton}
{
  retval = SUNNonlinSolGetSysFn\_Newton(NLS, SysFn);
}
{
  The function \ID{SUNNonlinSolGetSysFn\_Newton} returns the residual function
  that defines the nonlinear system.
}
{
  \begin{args}[SysFn]
  \item[NLS] (\id{SUNNonlinearSolver})
    a {\sunnonlinsol} object
  \item[SysFn] (\id{SUNNonlinSolSysFn*})
    the function defining the nonlinear system.
  \end{args}
}
{
  The return value \id{retval} (of type \id{int}) should be zero for a
  successful call, and a negative value for a failure.
}
{
  This function is intended for users that wish to evaluate the
  nonlinear residual in a custom convergence test function for the
  {\sunnonlinsolnewton} module.  We note that {\sunnonlinsolnewton}
  will not leverage the results from any user calls to \id{SysFn}.
}


% ====================================================================
\subsection{SUNNonlinearSolver\_Newton Fortran interfaces}
\label{ss:sunnonlinsolnewton_fortran}
% ====================================================================

The {\sunnonlinsolnewton} module provides a {\F} 2003 module as well as
{\F} 77 style interface functions for use from {\F} applications.

\subsubsection*{FORTRAN 2003 interface module}
The \ID{fsunnonlinsol\_newton\_mod} {\F} module defines interfaces to all
{\sunnonlinsolnewton} {\CC} functions using the intrinsic \id{iso\_c\_binding}
module which provides a standardized mechanism for interoperating with {\CC}. As
noted in the {\CC} function descriptions above, the interface functions are
named after the corresponding {\CC} function, but with a leading `F'. For
example, the function \id{SUNNonlinSol\_Newton} is interfaced as
\id{FSUNNonlinSol\_Newton}.

The {\F} 2003 {\sunnonlinsolnewton} interface module can be accessed with the \id{use}
statement, i.e. \id{use fsunnonlinsol\_newton\_mod}, and linking to the library \newline
\id{libsundials\_fsunnonlinsolnewton\_mod}.{\em lib} in addition to the {\CC}
library. For details on where the library and module file 
\id{fsunnonlinsol\_newton\_mod.mod} are installed see Appendix \ref{c:install}.
We note that the module is accessible from the {\F} 2003 {\sundials} integrators
\textit{without} separately linking to the
\id{libsundials\_fsunnonlinsolnewton\_mod} library.

\subsubsection*{FORTRAN 77 interface functions}
For {\sundials} integrators that include a {\F} 77 interface, the
{\sunnonlinsolnewton} module also includes a Fortran-callable
function for creating a \id{SUNNonlinearSolver} object.
%
% --------------------------------------------------------------------
%
\ucfunction{FSUNNEWTONINIT}
{
  FSUNNEWTONINIT(code, ier);
}
{
  The function \ID{FSUNNEWTONINIT} can be called for Fortran programs
  to create a\\
  \id{SUNNonlinearSolver} object for use with {\sundials}
  integrators to solve nonlinear systems of the form $F(y) = 0$ with
  Newton's method.
}
{
  \begin{args}[code]
  \item[code] (\id{int*})
    is an integer input specifying the solver id (1 for {\cvode}, 2
    for {\ida}, and 4 for {\arkode}).
  \end{args}
}
{
  \id{ier} is a return completion flag equal to \id{0} for a success
  return and \id{-1} otherwise. See printed message for details in case
  of failure.
}
{}


% ====================================================================
\subsection{SUNNonlinearSolver\_Newton content}
\label{ss:sunnonlinsolnewton_content}
% ====================================================================

The {\sunnonlinsolnewton} module defines the {\textit{content} field of a
\id{SUNNonlinearSolver} as the following structure:
%%
%%
\begin{verbatim}
struct _SUNNonlinearSolverContent_Newton {

  SUNNonlinSolSysFn      Sys;
  SUNNonlinSolLSetupFn   LSetup;
  SUNNonlinSolLSolveFn   LSolve;
  SUNNonlinSolConvTestFn CTest;

  N_Vector    delta;
  booleantype jcur;
  int         curiter;
  int         maxiters;
  long int    niters;
  long int    nconvfails;
  void*       ctest_data;
};
\end{verbatim}
%%
%%
These entries of the \emph{content} field contain the following
information:
\begin{args}[ctest\_data]
  \item[Sys]         - the function for evaluating the nonlinear system,
  \item[LSetup]      - the package-supplied function for setting up the linear solver,
  \item[LSolve]      - the package-supplied function for performing a linear solve,
  \item[CTest]       - the function for checking convergence of the Newton
                       iteration,
  \item[delta]       - the Newton iteration update vector,
  \item[jcur]        - the Jacobian status (\id{SUNTRUE} = current,
                       \id{SUNFALSE} = stale),
  \item[curiter]     - the current number of iterations in the solve attempt,
  \item[maxiters]    - the maximum number of Newton iterations allowed in
                       a solve,
  \item[niters]      - the total number of nonlinear iterations across all
                       solves,
  \item[nconvfails]  - the total number of nonlinear convergence failures across
                       all solves, and
  \item[ctest\_data] - the data pointer passed to the convergence test function.
\end{args}

% ====================================================================
\section{The SUNNonlinearSolver\_PetscSNES implementation}
\label{s:sunnonlinsolpetsc}
% ====================================================================
This section describes the {\sunnonlinsol} interface to the PETSc SNES nonlinear
solver(s). To enable the {\sunnonlinsolpetsc} module, SUNDIALS must be
configured to use PETSc. Instructions on how to do this are given in
Chapter~\ref{ss:building_with_petsc}. To access the module, users must include
the header file \id{sunnonlinsol/sunnonlinsol\_petscsnes.h}. The library to link
to is \id{libsundials\_sunnonlinsolpetsc}{\em .lib} where {\em .lib} is .so for
shared libaries and .a for static libraries. Users of the {\sunnonlinsolpetsc}
should also see the section {\nvecpetsc} \ref{ss:nvec_petsc} which discusses the
{\nvector} interface to the PETSc Vec API.

% ====================================================================
\subsection{SUNNonlinearSolver\_PetscSNES description}
\label{ss:sunnonlinsolpetsc_description}
% ====================================================================

The {\sunnonlinsolpetsc} implementation allows users to utilize a PETSc SNES
nonlinear solver to solve the nonlinear systems that arise in the {\sundials}
integrators. Since SNES uses the KSP linear solver interface underneath it, the
{\sunnonlinsolpetsc} implementation does not interface with {\sundials} linear
solvers. Instead, users should set nonlinear solver options, linear solver
options, and preconditioner options through the PETSc SNES, KSP, and PC APIs
\cite{petsc-user-ref}.

{\warn}

Important usage notes for the {\sunnonlinsolpetsc} implementation are provided
below:

\begin{itemize}
\item The {\sunnonlinsolpetsc} implementation handles calling \id{SNESSetFunction}
at construction. The actual residual function $F(y)$ is set by the {\sundials}
integrator when the {\sunnonlinsolpetsc} object is attached to it. Therefore, a
user should not call \id{SNESSetFunction} on a \id{SNES} object that is being
used with {\sunnonlinsolpetsc}. For these reasons, it is recommended, although
not always necessary, that the user calls \id{SUNNonlinSol\_PetscSNES} with the
new \id{SNES} object immediately after calling

\item The number of nonlinear iterations is tracked by {\sundials} separately
from the count kept by SNES. As such, the function \id{SUNNonlinSolGetNumIters}
reports the cumulative number of iterations across the lifetime of the
{\sunnonlinsol} object.

\item Some ``converged'' and ``diverged'' convergence reasons returned by SNES
are treated as recoverable convergence failures by {\sundials}. Therefore, the
count of convergence failures returned by \id{SUNNonlinSolGetNumConvFails} will
reflect the number of recoverable convergence failures as determined by
{\sundials}, and may differ from the count returned by
\id{SNESGetNonlinearStepFailures}.

\item The {\sunnonlinsolpetsc} module is not currently compatible with the
{\cvodes} or {\idas} staggered or simultaneous sensitivity strategies.
\end{itemize}

% ====================================================================
\subsection{SUNNonlinearSolver\_PetscSNES functions}
\label{ss:sunnonlinsolpetsc_functions}
% ====================================================================

The {\sunnonlinsolpetsc} module provides the following constructor
for creating a \id{SUNNonlinearSolver} object.
% --------------------------------------------------------------------
\ucfunction{SUNNonlinSol\_PetscSNES}
{
  NLS = SUNNonlinSol\_PetscSNES(y, snes);
}
{
  The function \ID{SUNNonlinSol\_PetscSNES} creates a \id{SUNNonlinearSolver}
  object that wraps a PETSc \id{SNES} object for use with {\sundials}.
  This will call \id{SNESSetFunction} on the provided \id{SNES} object.
}
{
  \begin{args}[snes]
  \item[snes] (\id{SNES})
    a PETSc \id{SNES} object
  \item[y] (\id{N\_Vector})
    a \id{N\_Vector} object of type {\nvecpetsc} that used as a template
    for the residual vector
  \end{args}
}
{
  A {\sunnonlinsol} object if the constructor exits successfully,
  otherwise \id{NLS} will be \id{NULL}.
}
{
  {\warn} This function calls \id{SNESSetFunction} and will overwrite
  whatever function was previously set. Users should not call
  \id{SNESSetFunction} on the \id{SNES} object provided to the constructor.
}

% --------------------------------------------------------------------
The {\sunnonlinsolpetsc} module implements all of the functions
defined in sections \ref{ss:sunnonlinsol_corefn} --
\ref{ss:sunnonlinsol_getfn} except for \id{SUNNonlinSolSetup},
\id{SUNNonlinSolSetLSetupFn},\newline \id{SUNNonlinSolSetLSolveFn},
\id{SUNNonlinSolSetConvTestFn}, and \id{SUNNonlinSolSetMaxIters}.

The {\sunnonlinsolpetsc} functions have the same names as those defined
by the generic {\sunnonlinsol} API with \id{\_PetscSNES} appended to the
function name. Unless using the {\sunnonlinsolpetsc} module as a
standalone nonlinear solver the generic functions defined in sections
\ref{ss:sunnonlinsol_corefn} -- \ref{ss:sunnonlinsol_getfn} should be
called in favor of the {\sunnonlinsolpetsc}-specific implementations.

The {\sunnonlinsolpetsc} module also defines the following additional
user-callable functions.
% --------------------------------------------------------------------
\ucfunction{SUNNonlinSolGetSNES\_PetscSNES}
{
  retval = SUNNonlinSolGetSNES\_PetscSNES(NLS, SNES* snes);
}
{
  The function \ID{SUNNonlinSolGetSNES\_PetscSNES} gets the
  SNES context that was wrapped.
}
{
  \begin{args}[SysFn]
  \item[NLS] (\id{SUNNonlinearSolver})
    a {\sunnonlinsol} object
  \item[snes] (\id{SNES*})
    a pointer to a PETSc \id{SNES} object that will be set upon return
  \end{args}
}
{
  The return value \id{retval} (of type \id{int}) should be zero for a
  successful call, and a negative value for a failure.
}
{}
% --------------------------------------------------------------------
\ucfunction{SUNNonlinSolGetPetscError\_PetscSNES}
{
  retval = SUNNonlinSolGetPetscError\_PetscSNES(NLS, PestcErrorCode* error);
}
{
  The function \ID{SUNNonlinSolGetPetscError\_PetscSNES} gets the last error
  code returned by the last internal call to a PETSc API function.
}
{
  \begin{args}[SysFn]
  \item[NLS] (\id{SUNNonlinearSolver})
    a {\sunnonlinsol} object
  \item[error] (\id{PestcErrorCode*})
    a pointer to a PETSc error integer that will be set upon return
  \end{args}
}
{
  The return value \id{retval} (of type \id{int}) should be zero for a
  successful call, and a negative value for a failure.
}
{}
% --------------------------------------------------------------------
\ucfunction{SUNNonlinSolGetSysFn\_PetscSNES}
{
  retval = SUNNonlinSolGetSysFn\_PetscSNES(NLS, SysFn);
}
{
  The function \ID{SUNNonlinSolGetSysFn\_PetscSNES} returns the residual
  function that defines the nonlinear system.
}
{
  \begin{args}[SysFn]
  \item[NLS] (\id{SUNNonlinearSolver})
    a {\sunnonlinsol} object
  \item[SysFn] (\id{SUNNonlinSolSysFn*})
    the function defining the nonlinear system
  \end{args}
}
{
  The return value \id{retval} (of type \id{int}) should be zero for a
  successful call, and a negative value for a failure.
}
{}

% ====================================================================
\subsection{SUNNonlinearSolver\_PetscSNES content}
\label{ss:sunnonlinsolpetsc_content}
% ====================================================================

The {\sunnonlinsolpetsc} module defines the {\textit{content} field of a
\id{SUNNonlinearSolver} as the following structure:
%%
%%
\begin{verbatim}
struct _SUNNonlinearSolverContent_PetscSNES {
  int sysfn_last_err;
  PetscErrorCode petsc_last_err;
  long int nconvfails;
  long int nni;
  void *imem;
  SNES snes;
  Vec r;
  N_Vector y, f;
  SUNNonlinSolSysFn Sys;
};
\end{verbatim}
%%
%%
These entries of the \emph{content} field contain the following
information:
\begin{args}[Sys]
  \item[sysfn\_last\_err]  - last error returned by the system defining function,
  \item[petsc\_last\_err]  - last error returned by PETSc
  \item[nconvfails]        - number of nonlinear converge failures (recoverable or not),
  \item[nni]               - number of nonlinear iterations,
  \item[imem]              - {\sundials} integrator memory,
  \item[snes]              - PETSc SNES context,
  \item[r]                 - the nonlinear residual,
  \item[y]                 - wrapper for PETSc vectors used in the system function,
  \item[f]                 - wrapper for PETSc vectors used in the system function,
  \item[Sys]               - nonlinear system definining function.
\end{args}

