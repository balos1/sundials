%% This is a shared SUNDIALS TEX file with a description of the
%% OpenMPDEV nvector implementation
%%
\section{The NVECTOR\_OPENMPDEV implementation}\label{ss:nvec_openmpdev}

In situations where a user has access to a device such as a GPU for
offloading computation, {\sundials} provides an {\nvector} implementation using
OpenMP device offloading, called {\nvecopenmpdev}.

The {\nvecopenmpdev} implementation defines the \textit{content} field
of the \id{N\_Vector} to be a structure  containing the length of the vector, a pointer
to the beginning of a contiguous  data array on the host, a pointer to the beginning of
a contiguous data array on the device, and a boolean flag \id{own\_data} which specifies
the ownership of host and device data arrays.
%%
\begin{verbatim}
struct _N_VectorContent_OpenMPDEV {
  sunindextype length;
  booleantype own_data;
  realtype *host_data;
  realtype *dev_data;
};
\end{verbatim}
%%
%%--------------------------------------------

The header file to include when using this module is \id{nvector\_openmpdev.h}.
The installed module library to link to is
\id{libsundials\_nvecopenmpdev.\textit{lib}}
where \id{\em.lib} is typically \id{.so} for shared libraries and \id{.a}
for static libraries.


% ====================================================================
\subsection{NVECTOR\_OPENMPDEV accessor macros}
\label{ss:nvec_openmpdev_macros}
% ====================================================================

The following macros are provided to access the content of an {\nvecopenmpdev}
vector.
%%
\begin{itemize}

\item \ID{NV\_CONTENT\_OMPDEV}

  This routine gives access to the contents of the {\nvecopenmpdev}
  vector \id{N\_Vector}.

  The assignment \id{v\_cont} $=$ \id{NV\_CONTENT\_OMPDEV(v)} sets
  \id{v\_cont} to be a pointer to the {\nvecopenmpdev} \id{N\_Vector} content
  structure.

  Implementation:

  \verb|#define NV_CONTENT_OMPDEV(v) ( (N_VectorContent_OpenMPDEV)(v->content) )|

\item \ID{NV\_OWN\_DATA\_OMPDEV}, \ID{NV\_DATA\_HOST\_OMPDEV}, \ID{NV\_DATA\_DEV\_OMPDEV}, \ID{NV\_LENGTH\_OMPDEV}

  These macros give individual access to the parts of
  the content of an {\nvecopenmpdev} \id{N\_Vector}.

  The assignment \id{v\_data = NV\_DATA\_HOST\_OMPDEV(v)} sets \id{v\_data} to be
  a pointer to the first component of the data on the host for the \id{N\_Vector} \id{v}.
  The assignment \id{NV\_DATA\_HOST\_OMPDEV(v) = v\_data} sets the host component array of \id{v} to
  be \id{v\_data} by storing the pointer \id{v\_data}.

  The assignment \id{v\_dev\_data = NV\_DATA\_DEV\_OMPDEV(v)} sets \id{v\_dev\_data} to be
  a pointer to the first component of the data on the device for the \id{N\_Vector} \id{v}.
  The assignment \id{NV\_DATA\_DEV\_OMPDEV(v) = v\_dev\_data} sets the device component array of \id{v} to
  be \id{v\_dev\_data} by storing the pointer \id{v\_dev\_data}.

  The assignment \id{v\_len = NV\_LENGTH\_OMPDEV(v)} sets \id{v\_len} to be
  the length of \id{v}. On the other hand, the call \id{NV\_LENGTH\_OMPDEV(v) = len\_v}
  sets the length of \id{v} to be \id{len\_v}.

  Implementation:

  \verb|#define NV_OWN_DATA_OMPDEV(v) ( NV_CONTENT_OMPDEV(v)->own_data )|

  \verb|#define NV_DATA_HOST_OMPDEV(v) ( NV_CONTENT_OMPDEV(v)->host_data )|

  \verb|#define NV_DATA_DEV_OMPDEV(v) ( NV_CONTENT_OMPDEV(v)->dev_data )|

  \verb|#define NV_LENGTH_OMPDEV(v) ( NV_CONTENT_OMPDEV(v)->length )|

\end{itemize}


% ====================================================================
\subsection{NVECTOR\_OPENMPDEV functions}
\label{ss:nvec_openmpdev_functions}
% ====================================================================%

The {\nvecopenmpdev} module defines OpenMP device offloading implementations of
all vector operations listed in Tables \ref{ss:nvecops}, \ref{ss:nvecfusedops},
\ref{ss:nvecarrayops}, and \ref{ss:nveclocalops}, except for \id{N\_VGetArrayPointer} and
\id{N\_VSetArrayPointer}. As such, this vector cannot be used with the
{\sundials} Fortran interfaces, nor with the {\sundials} direct solvers and
preconditioners. It also provides methods for copying from the host to
the device and vice versa.

The names of vector operations are obtained from those in Tables
\ref{ss:nvecops}, \ref{ss:nvecfusedops}, \ref{ss:nvecarrayops}, and
\ref{ss:nveclocalops} by appending the
suffix \id{\_OpenMPDEV} (e.g. \id{N\_VDestroy\_OpenMPDEV}). The module
{\nvecopenmpdev} provides the following additional user-callable routines:
%%--------------------------------------
\sunmodfun{N\_VNew\_OpenMPDEV}
{
  This function creates and allocates memory for an {\nvecopenmpdev} \id{N\_Vector}.
}
{
  N\_Vector N\_VNew\_OpenMPDEV(sunindextype vec\_length)
}
%%--------------------------------------
\sunmodfun{N\_VNewEmpty\_OpenMPDEV}
{
  This function creates a new {\nvecopenmpdev} \id{N\_Vector} with an empty
  (\id{NULL}) host and device data arrays.
}
{
  N\_Vector N\_VNewEmpty\_OpenMPDEV(sunindextype vec\_length)
}
%%--------------------------------------
\sunmodfun{N\_VMake\_OpenMPDEV}
{
 This function creates an {\nvecopenmpdev} vector with user-supplied vector data
 arrays \id{h\_vdata} and \id{d\_vdata}. This function does not allocate memory for
 data itself.
}
{
  N\_Vector N\_VMake\_OpenMPDEV(sunindextype vec\_length, realtype *h\_vdata,
  \newlinefill{N\_Vector N\_VMake\_OpenMPDEV}
  realtype *d\_vdata)
}
%%--------------------------------------
\sunmodfun{N\_VCloneVectorArray\_OpenMPDEV}
{
 This function creates (by cloning) an array of \id{count} {\nvecopenmpdev} vectors.
}
{
 N\_Vector *N\_VCloneVectorArray\_OpenMPDEV(int count, N\_Vector w)
}
%%--------------------------------------
\sunmodfun{N\_VCloneVectorArrayEmpty\_OpenMPDEV}
{
 This function creates (by cloning) an array of \id{count} {\nvecopenmpdev} vectors, each with an
 empty (\id{NULL}) data array.
}
{
 N\_Vector *N\_VCloneVectorArrayEmpty\_OpenMPDEV(int count, N\_Vector w)
}
%%--------------------------------------
\sunmodfun{N\_VDestroyVectorArray\_OpenMPDEV}
{
 This function frees memory allocated for the array of \id{count} variables of type
 \id{N\_Vector} created with \id{N\_VCloneVectorArray\_OpenMPDEV} or with \newline
 \id{N\_VCloneVectorArrayEmpty\_OpenMPDEV}.
}
{
 void N\_VDestroyVectorArray\_OpenMPDEV(N\_Vector *vs, int count)
}
%%--------------------------------------
\sunmodfun{N\_VGetHostArrayPointer\_OpenMPDEV}
{
 This function returns a pointer to the host data array.
}
{
 realtype *N\_VGetHostArrayPointer\_OpenMPDEV(N\_Vector v)
}
%%--------------------------------------
\sunmodfun{N\_VGetDeviceArrayPointer\_OpenMPDEV}
{
 This function returns a pointer to the device data array.
}
{
 realtype *N\_VGetDeviceArrayPointer\_OpenMPDEV(N\_Vector v)
}
%%--------------------------------------
\sunmodfun{N\_VPrint\_OpenMPDEV}
{
 This function prints the content of an {\nvecopenmpdev} vector to \id{stdout}.
}
{
 void N\_VPrint\_OpenMPDEV(N\_Vector v)
}
%%--------------------------------------
\sunmodfun{N\_VPrintFile\_OpenMPDEV}
{
 This function prints the content of an {\nvecopenmpdev} vector to \id{outfile}.
}
{
 void N\_VPrintFile\_OpenMPDEV(N\_Vector v, FILE *outfile)
}
%%--------------------------------------
\sunmodfun{N\_VCopyToDevice\_OpenMPDEV}
{
 This function copies the content of an {\nvecopenmpdev} vector's host data array
 to the device data array.
}
{
 void N\_VCopyToDevice\_OpenMPDEV(N\_Vector v)
}
%%--------------------------------------
\sunmodfun{N\_VCopyFromDevice\_OpenMPDEV}
{
 This function copies the content of an {\nvecopenmpdev} vector's device data array
 to the host data array.
}
{
  void N\_VCopyFromDevice\_OpenMPDEV(N\_Vector v)
}
%%--------------------------------------

By default all fused and vector array operations are disabled in the {\nvecopenmpdev}
module. The following additional user-callable routines are provided to
enable or disable fused and vector array operations for a specific vector. To
ensure consistency across vectors it is recommended to first create a vector
with \id{N\_VNew\_OpenMPDEV}, enable/disable the desired operations for that vector
with the functions below, and create any additional vectors from that vector
using \id{N\_VClone}. This guarantees the new vectors will have the same
operations enabled/disabled as cloned vectors inherit the same enable/disable
options as the vector they are cloned from while vectors created with
\id{N\_VNew\_OpenMPDEV} will have the default settings for the {\nvecopenmpdev} module.
%%--------------------------------------
\sunmodfun{N\_VEnableFusedOps\_OpenMPDEV}
{
  This function enables (\id{SUNTRUE}) or disables (\id{SUNFALSE}) all fused and
  vector array operations in the {\nvecopenmpdev} vector. The return value is \id{0} for
  success and \id{-1} if the input vector or its \id{ops} structure are \id{NULL}.
}
{
  int N\_VEnableFusedOps\_OpenMPDEV(N\_Vector v, booleantype tf)
}
%%--------------------------------------
\sunmodfun{N\_VEnableLinearCombination\_OpenMPDEV}
{
  This function enables (\id{SUNTRUE}) or disables (\id{SUNFALSE}) the linear
  combination fused operation in the {\nvecopenmpdev} vector. The return value is \id{0} for
  success and \id{-1} if the input vector or its \id{ops} structure are \id{NULL}.
}
{
  int N\_VEnableLinearCombination\_OpenMPDEV(N\_Vector v, booleantype tf)
}
%%--------------------------------------
\sunmodfun{N\_VEnableScaleAddMulti\_OpenMPDEV}
{
  This function enables (\id{SUNTRUE}) or disables (\id{SUNFALSE}) the scale and
  add a vector to multiple vectors fused operation in the {\nvecopenmpdev} vector. The
  return value is \id{0} for success and \id{-1} if the input vector or its
  \id{ops} structure are \id{NULL}.
}
{
  int N\_VEnableScaleAddMulti\_OpenMPDEV(N\_Vector v, booleantype tf)
}
%%--------------------------------------
\sunmodfun{N\_VEnableDotProdMulti\_OpenMPDEV}
{
  This function enables (\id{SUNTRUE}) or disables (\id{SUNFALSE}) the multiple
  dot products fused operation in the {\nvecopenmpdev} vector. The return value is \id{0}
  for success and \id{-1} if the input vector or its \id{ops} structure are
  \id{NULL}.
}
{
  int N\_VEnableDotProdMulti\_OpenMPDEV(N\_Vector v, booleantype tf)
}
%%--------------------------------------
\sunmodfun{N\_VEnableLinearSumVectorArray\_OpenMPDEV}
{
  This function enables (\id{SUNTRUE}) or disables (\id{SUNFALSE}) the linear sum
  operation for vector arrays in the {\nvecopenmpdev} vector. The return value is \id{0} for
  success and \id{-1} if the input vector or its \id{ops} structure are \id{NULL}.
}
{
  int N\_VEnableLinearSumVectorArray\_OpenMPDEV(N\_Vector v, booleantype tf)
}
%%--------------------------------------
\sunmodfun{N\_VEnableScaleVectorArray\_OpenMPDEV}
{
  This function enables (\id{SUNTRUE}) or disables (\id{SUNFALSE}) the scale
  operation for vector arrays in the {\nvecopenmpdev} vector. The return value is \id{0} for
  success and \id{-1} if the input vector or its \id{ops} structure are \id{NULL}.
}
{
  int N\_VEnableScaleVectorArray\_OpenMPDEV(N\_Vector v, booleantype tf)
}
%%--------------------------------------
\sunmodfun{N\_VEnableConstVectorArray\_OpenMPDEV}
{
  This function enables (\id{SUNTRUE}) or disables (\id{SUNFALSE}) the const
  operation for vector arrays in the {\nvecopenmpdev} vector. The return value is \id{0} for
  success and \id{-1} if the input vector or its \id{ops} structure are \id{NULL}.
}
{
  int N\_VEnableConstVectorArray\_OpenMPDEV(N\_Vector v, booleantype tf)
}
%%--------------------------------------
\sunmodfun{N\_VEnableWrmsNormVectorArray\_OpenMPDEV}
{
  This function enables (\id{SUNTRUE}) or disables (\id{SUNFALSE}) the WRMS norm
  operation for vector arrays in the {\nvecopenmpdev} vector. The return value is \id{0} for
  success and \id{-1} if the input vector or its \id{ops} structure are \id{NULL}.
}
{
  int N\_VEnableWrmsNormVectorArray\_OpenMPDEV(N\_Vector v, booleantype tf)
}
%%--------------------------------------
\sunmodfun{N\_VEnableWrmsNormMaskVectorArray\_OpenMPDEV}
{
  This function enables (\id{SUNTRUE}) or disables (\id{SUNFALSE}) the masked WRMS
  norm operation for vector arrays in the {\nvecopenmpdev} vector. The return value is
  \id{0} for success and \id{-1} if the input vector or its \id{ops} structure are
  \id{NULL}.
}
{
  int N\_VEnableWrmsNormMaskVectorArray\_OpenMPDEV(N\_Vector v,
  \newlinefill{int N\_VEnableWrmsNormMaskVectorArray\_OpenMPDEV}
  booleantype tf)
}
%%--------------------------------------
\sunmodfun{N\_VEnableScaleAddMultiVectorArray\_OpenMPDEV}
{
  This function enables (\id{SUNTRUE}) or disables (\id{SUNFALSE}) the scale and
  add a vector array to multiple vector arrays operation in the {\nvecopenmpdev} vector. The
  return value is \id{0} for success and \id{-1} if the input vector or its
  \id{ops} structure are \id{NULL}.
}
{
  int N\_VEnableScaleAddMultiVectorArray\_OpenMPDEV(N\_Vector v,
  \newlinefill{int N\_VEnableScaleAddMultiVectorArray\_OpenMPDEV}
  booleantype tf)
}
%%--------------------------------------
\sunmodfun{N\_VEnableLinearCombinationVectorArray\_OpenMPDEV}
{
  This function enables (\id{SUNTRUE}) or disables (\id{SUNFALSE}) the linear
  combination operation for vector arrays in the {\nvecopenmpdev} vector. The return value
  is \id{0} for success and \id{-1} if the input vector or its \id{ops} structure
  are \id{NULL}.
}
{
  int N\_VEnableLinearCombinationVectorArray\_OpenMPDEV(N\_Vector v,
  \newlinefill{int N\_VEnableLinearCombinationVectorArray\_OpenMPDEV}
  booleantype tf)
}
%%
%%------------------------------------
%%
\paragraph{\bf Notes}

\begin{itemize}

\item
  When looping over the components of an \id{N\_Vector} \id{v}, it is
  most efficient to first obtain the component array via
  \id{h\_data = NV\_DATA\_HOST\_OMPDEV(v)} for the host array or \newline
  \id{d\_data = NV\_DATA\_DEV\_OMPDEV(v)} for the device array and then access
  \id{h\_data[i]} or \id{d\_data[i]} within the loop.

\item
  When accessing individual components of an \id{N\_Vector} \id{v} on
  the host remember to first copy the array
  back from the device with \id{N\_VCopyFromDevice\_OpenMPDEV(v)}
  to ensure the array is up to date.

\item
  {\warn}\id{N\_VNewEmpty\_OpenMPDEV}, \id{N\_VMake\_OpenMPDEV},
  and \id{N\_VCloneVectorArrayEmpty\_OpenMPDEV} set the field
  \id{own\_data} $=$ \id{SUNFALSE}.
  \id{N\_VDestroy\_OpenMPDEV} and \id{N\_VDestroyVectorArray\_OpenMPDEV}
  will not attempt to free the pointer {\em data} for any \id{N\_Vector} with
  \id{own\_data} set to \id{SUNFALSE}. In such a case, it is the user's responsibility to
  deallocate the {\em data} pointer.

\item
  {\warn}To maximize efficiency, vector operations in the {\nvecopenmpdev} implementation
  that have more than one \id{N\_Vector} argument do not check for
  consistent internal representation of these vectors. It is the user's
  responsibility to ensure that such routines are called with \id{N\_Vector}
  arguments that were all created with the same internal representations.

\end{itemize}
