% This is a shared SUNDIALS TEX file with description of
% the generic sunmatrix abstraction
%
For problems that involve direct methods for solving linear systems,
the {\sundials} solvers not only operate on generic vectors, but also
on generic matrices (of type \Id{SUNMatrix}), through a set of
operations defined by the particular {\sunmatrix} implementation.
Users can provide their own specific implementation of the
{\sunmatrix} module, particularly in cases where they provide their
own {\nvector} and/or linear solver modules, and require matrices that
are compatible with those implementations.  Alternately, we provide three
{\sunmatrix} implementations: dense, banded, and sparse.  The
generic operations are described below, and descriptions of the
implementations provided with {\sundials} follow.

% ====================================================================
\section{The SUNMatrix API}
\label{s:sunmatrix_api}
% ====================================================================

The {\sunmatrix} API can be grouped into two sets of functions:
the core matrix operations, and utility functions. Section \ref{ss:sunmatrix_functions}
lists the core operations, while Section \ref{ss:sunmatrix_utilities} lists
the utility functions.

%==============================================================================
\subsection{SUNMatrix core functions}\label{ss:sunmatrix_functions}

The generic \id{SUNMatrix} object defines the following set of core operations:

\ucfunctionf{SUNMatGetID}
{
  id = SUNMatGetID(A);
}
{
  Returns the type identifier for the matrix \id{A}. It is used to determine the
  matrix implementation type (e.g.~dense, banded, sparse,\ldots) from the abstract
  \id{SUNMatrix} interface.  This is used to assess compatibility with
  {\sundials}-provided linear solver implementations.
}
{
  \begin{args}[c]
  \item[A] (\id{SUNMatrix}) a {\sunmatrix} object
  \end{args}
}
{
  A \id{SUNMATRIX\_ID}, possible values are given in the Table \ref{t:matrixIDs}.
}
{}

\ucfunctionfl{SUNMatClone}
{
  B = SUNMatClone(A);
}
{
  Creates a new \id{SUNMatrix} of the same type as an existing matrix
  \id{A} and sets the {\em ops} field. It does not copy the matrix, but
  rather allocates storage for the new matrix.
}
{
  \begin{args}[c]
  \item[A] (\id{SUNMatrix}) a {\sunmatrix} object
  \end{args}
}
{
  \id{SUNMatrix}
}
{}
{
  type(SUNMatrix), pointer :: B\\
  B => FSUNMatClone(A)
}

\ucfunctionf{SUNMatDestroy}
{
  SUNMatDestroy(A);
}
{
  Destroys \id{A} and frees memory allocated for its internal data.
}
{
  \begin{args}[c]
  \item[A] (\id{SUNMatrix}) a {\sunmatrix} object
  \end{args}
}
{}
{}

\ucfunctionfl{SUNMatSpace}
{
  ier = SUNMatSpace(A, \&lrw, \&liw);
}
{
  Returns the storage requirements for the matrix \id{A}. \id{lrw}
  is a \id{long int} containing the number of realtype words
  and \id{liw} is a \id{long int} containing the number of integer
  words.
}
{
  \begin{args}[c]
  \item[A] (\id{SUNMatrix}) a {\sunmatrix} object
  \item[lrw] (\id{sunindextype*}) the number of realtype words
  \item[liw] (\id{sunindextype*}) the number of integer words
  \end{args}
}
{}
{
  This function is advisory only, for use in determining a user's total
  space requirements; it could be a dummy function in a user-supplied
  {\sunmatrix} module if that information is not of interest.
}
{
  integer(c\_long) :: lrw(1), liw(1)\\
  ier = FSUNMatSpace(A, lrw, liw)
}

\ucfunctionf{SUNMatZero}
{
  ier = SUNMatZero(A);
}
{
  Performs the operation $A_{ij} = 0$ for all entries of the matrix $A$.
}
{
  \begin{args}[c]
  \item[A] (\id{SUNMatrix}) a {\sunmatrix} object
  \end{args}
}
{
  A {\sunmatrix} return code of type \id{int} denoting success/failure
}
{}

\ucfunctionf{SUNMatCopy}
{
  ier = SUNMatCopy(A,B);
}
{
  Performs the operation $B_{ij} = A_{i,j}$ for all entries of the matrices
  $A$ and $B$.
}
{
  \begin{args}[c]
  \item[A] (\id{SUNMatrix}) a {\sunmatrix} object
  \item[B] (\id{SUNMatrix}) a {\sunmatrix} object
  \end{args}
}
{
  A {\sunmatrix} return code of type \id{int} denoting success/failure
}
{}

\ucfunctionf{SUNMatScaleAdd}
{
  ier = SUNMatScaleAdd(c, A, B);
}
{
  Performs the operation $A = cA + B$.
}
{
  \begin{args}[c]
  \item[c] (\id{realtype}) constant that scales \id{A}
  \item[A] (\id{SUNMatrix}) a {\sunmatrix} object
  \item[B] (\id{SUNMatrix}) a {\sunmatrix} object
  \end{args}
}
{
  A {\sunmatrix} return code of type \id{int} denoting success/failure
}
{}

\ucfunctionf{SUNMatScaleAddI}
{
  ier = SUNMatScaleAddI(c, A);
}
{
  Performs the operation $A = cA + I$.
}
{
  \begin{args}[c]
  \item[c] (\id{realtype}) constant that scales \id{A}
  \item[A] (\id{SUNMatrix}) a {\sunmatrix} object
  \end{args}
}
{
  A {\sunmatrix} return code of type \id{int} denoting success/failure
}
{}

\ucfunctionf{SUNMatMatvecSetup}
{
  ier = SUNMatMatvecSetup(A);
}
{
  Performs any setup necessary to perform a matrix-vector product.
  It is useful for SUNMatrix implementations which need to prepare
  the matrix itself, or communication structures before performing
  the matrix-vector product.
}
{
  \begin{args}[A]
  \item[A] (\id{SUNMatrix}) a {\sunmatrix} object
  \end{args}
}
{
  A {\sunmatrix} return code of type \id{int} denoting success/failure
}
{}

\ucfunctionf{SUNMatMatvec}
{
  ier = SUNMatMatvec(A, x, y);
}
{
  Performs the matrix-vector product operation, $y = Ax$. It should
  only be called with vectors \id{x} and \id{y} that are compatible with
  the matrix \id{A} -- both in storage type and dimensions.
}
{
  \begin{args}[A]
  \item[A] (\id{SUNMatrix}) a {\sunmatrix} object
  \item[x] (\id{N\_Vector}) a {\nvector} object
  \item{y} (\id{N\_Vector}) an output {\nvector} object
  \end{args}
}
{
  A {\sunmatrix} return code of type \id{int} denoting success/failure
}
{}


%==============================================================================
\subsection{SUNMatrix utility functions}\label{ss:sunmatrix_utilities}

To aid in the creation of custom {\sunmatrix} modules the generic {\sunmatrix}
module provides two utility functions \id{SUNMatNewEmpty} and
\id{SUNMatVCopyOps}.

\ucfunctionf{SUNMatNewEmpty}
{
  A = SUNMatNewEmpty();
}
{
  The function \Id{SUNMatNewEmpty} allocates a new generic {\sunmatrix} object
  and initializes its content pointer and the function pointers in the
  operations structure to \id{NULL}.
}
{}
{
  This function returns a \id{SUNMatrix} object. If an error occurs when
  allocating the object, then this routine will return \id{NULL}.
}
{}

\ucfunctionf{SUNMatFreeEmpty}
{
  SUNMatFreeEmpty(A);
}
{
  This routine frees the generic \id{SUNMatrix} object, under the assumption that any
  implementation-specific data that was allocated within the underlying content structure
  has already been freed. It will additionally test whether the ops pointer is \id{NULL},
  and, if it is not, it will free it as well.
}
{
  \begin{args}[A]
  \item[A] (\id{SUNMatrix}) a \id{SUNMatrix} object
  \end{args}
}
{}
{}

\ucfunctionf{SUNMatCopyOps}
{
  retval = SUNMatCopyOps(A, B);
}
{
  The function \Id{SUNMatCopyOps} copies the function pointers in the \id{ops}
  structure of \id{A} into the \id{ops} structure of \id{B}.
}
{
  \begin{args}[A]
  \item[A] (\id{SUNMatrix}) the matrix to copy operations from
  \item[B] (\id{SUNMatrix}) the matrix to copy operations to
  \end{args}
}
{
  This returns \id{0} if successful and a non-zero value if either of the inputs
  are \id{NULL} or the \id{ops} structure of either input is \id{NULL}.
}
{}


%==============================================================================
\subsection{SUNMatrix return codes}\label{ss:sunmatrix_ReturnCodes}

The functions provided to {\sunmatrix} modules within the
{\sundials}-provided {\sunmatrix} implementations utilize a common
set of return codes, shown in Table \ref{t:sunmatrixerr}. These adhere
to a common pattern: 0 indicates success, and a negative value
indicates a failure. The actual values of each return code are
primarily to provide additional information to the user in case of
a failure.

\newlength{\AColOne}
\settowidth{\AColOne}{\id{SUNMAT\_MATVEC\_SETUP\_REQUIRED}}
\newlength{\AColTwo}
\settowidth{\AColTwo}{\id{Value}}
\newlength{\AColThree}
\setlength{\AColThree}{\textwidth}
\addtolength{\AColThree}{-0.5in}
\addtolength{\AColThree}{-\AColOne}
\addtolength{\AColThree}{-\AColTwo}

\tablecaption{Description of the \id{SUNMatrix} return codes}\label{t:sunmatrixerr}
\tablehead{\hline {\rule{0mm}{5mm}}{\bf Name} & {\bf Value} & {\bf Description} \\[3mm] \hline\hline}
\tabletail{\hline \multicolumn{3}{|r|}{\small\slshape continued on next page} \\ \hline}
\begin{xtabular}{|p{\AColOne}|p{\AColTwo}|p{\AColThree}|}
%%
\id{SUNMAT\_SUCCESS} & \id{0} & successful call or converged solve
\\[1mm]
%%
\id{SUNMAT\_ILL\_INPUT} & \id{-701} & an illegal input has been provided to the function
\\[1mm]
%%
\id{SUNMAT\_MEM\_FAIL} & \id{-702} & failed memory access or allocation
\\[1mm]
%%
\id{SUNMAT\_OPERATION\_FAIL} & \id{-703} & a SUNMatrix operation returned nonzero
\\
%%
\id{SUNMAT\_MATVEC\_SETUP\_REQUIRED} & \id{-704} & the \id{SUNMatMatvecSetup} routine
needs to be called before calling \id{SUNMatMatvec}
\\
\hline
\end{xtabular}
\bigskip


%==============================================================================
\subsection{SUNMatrix identifiers}\label{ss:sunmatrix_identifiers}

Each {\sunmatrix} implementation included in {\sundials} has a unique
identifier specified in enumeration and shown in Table \ref{t:matrixIDs}.
It is recommended that a user-supplied {\sunmatrix} implementation use the
\id{SUNMATRIX\_CUSTOM} identifier.

\begin{table}
\centering
\caption{Identifiers associated with matrix kernels supplied with {\sundials}.}
\label{t:matrixIDs}
\medskip
\begin{tabular}{|l|l|c|}
\hline
{\bf Matrix ID} & {\bf Matrix type} & {\bf ID Value} \\
\hline
SUNMATRIX\_DENSE      & Dense $\id{M} \times \id{N}$ matrix               & 0 \\
SUNMATRIX\_BAND       & Band $\id{M} \times \id{M}$ matrix                & 1 \\
SUNMATRIX\_MAGMADENSE & Magma dense $\id{M} \times \id{N}$ matrix         & 2 \\
SUNMATRIX\_SPARSE     & Sparse (CSR or CSC) $\id{M} \times \id{N}$ matrix & 3 \\
SUNMATRIX\_SLUNRLOC   & Adapter for the {\superludist} \id{SuperMatrix}   & 4 \\
SUNMATRIX\_CUSPARSE   & CUDA sparse CSR matrix                            & 5 \\
SUNMATRIX\_CUSTOM     & User-provided custom matrix                       & 6 \\
\hline
\end{tabular}
\end{table}


%==============================================================================
\subsection{Compatibility of SUNMatrix modules}\label{ss:sunmatrix_compatibility}

We note that not all {\sunmatrix} types are compatible with all
{\nvector} types provided with {\sundials}.  This is primarily due to
the need for compatibility within the \id{SUNMatMatvec} routine;
however, compatibility between {\sunmatrix} and {\nvector}
implementations is more crucial when considering their interaction
within {\sunlinsol} objects, as will be described in more detail in
Chapter \ref{s:sunlinsol}.  More specifically, in Table
\ref{t:matrix-vector} we show the matrix interfaces available as
{\sunmatrix} modules, and the compatible vector implementations.

\tablecaption{{\sundials} matrix interfaces and vector
             implementations that can be used for each.}\label{t:matrix-vector}
\tablehead{\hline \multicolumn{1}{|p{1.5cm}|}{{Matrix Interface}} &
                  \multicolumn{1}{p{0.7cm}|}{{Serial}} &
                  \multicolumn{1}{p{1.1cm}|}{{Parallel (MPI)}} &
                  \multicolumn{1}{p{1.3cm}|}{{OpenMP}} &
                  \multicolumn{1}{p{1.3cm}|}{{pThreads}} &
                  \multicolumn{1}{p{0.9cm}|}{{{\hypre} Vec.}} &
                  \multicolumn{1}{p{0.9cm}|}{{{\petsc} Vec.}} &
                  \multicolumn{1}{p{0.8cm}|}{{{\cuda}}} &
                  \multicolumn{1}{p{0.8cm}|}{{{\raja}}} &
                  \multicolumn{1}{p{1.1cm}|}{{User Suppl.}} \\ \hline }
\tabletail{\hline \multicolumn{10}{|r|}{\small\slshape continued on next page} \\ \hline}
{\renewcommand{\arraystretch}{1.2}
\begin{xtabular}{|l|c|c|c|c|c|c|c|c|c|}
    Dense         &  \cm     &           & \cm      &  \cm       &             &          &          &          & \cm      \\
    Band          &  \cm     &           & \cm      &  \cm       &             &          &          &          & \cm      \\
    Sparse        &  \cm     &           & \cm      &  \cm       &             &          &          &          & \cm      \\
    SLUNRloc      &  \cm     & \cm       & \cm      &  \cm       & \cm         &  \cm     &          &          & \cm      \\
    User supplied &  \cm     & \cm       & \cm      &  \cm       & \cm         &  \cm     & \cm      & \cm      & \cm      \\
    \hline
\end{xtabular}
}
\bigskip


\subsection{The generic SUNMatrix module implementation}\label{ss:sunmatrix_implmentation}

The generic \ID{SUNMatrix} type has been modeled after the
object-oriented style of the generic \id{N\_Vector} type.
Specifically, a generic \ID{SUNMatrix} is a pointer to a structure
that has an implementation-dependent {\em content} field containing
the description and actual data of the matrix, and an {\em ops} field
pointing to a structure with generic matrix operations.
The type \id{SUNMatrix} is defined as
%%
%%
\begin{verbatim}
typedef struct _generic_SUNMatrix *SUNMatrix;

struct _generic_SUNMatrix {
    void *content;
    struct _generic_SUNMatrix_Ops *ops;
};
\end{verbatim}
%%
%%
The \id{\_generic\_SUNMatrix\_Ops} structure is essentially a list of pointers to
the various actual matrix operations, and is defined as
%%
\begin{verbatim}
struct _generic_SUNMatrix_Ops {
  SUNMatrix_ID (*getid)(SUNMatrix);
  SUNMatrix    (*clone)(SUNMatrix);
  void         (*destroy)(SUNMatrix);
  int          (*zero)(SUNMatrix);
  int          (*copy)(SUNMatrix, SUNMatrix);
  int          (*scaleadd)(realtype, SUNMatrix, SUNMatrix);
  int          (*scaleaddi)(realtype, SUNMatrix);
  int          (*matvecsetup)(SUNMatrix)
  int          (*matvec)(SUNMatrix, N_Vector, N_Vector);
  int          (*space)(SUNMatrix, long int*, long int*);
};
\end{verbatim}


The generic {\sunmatrix} module defines and implements the matrix operations
acting on \id{SUNMatrix} objects.
These routines are nothing but wrappers for the matrix operations defined by
a particular {\sunmatrix} implementation, which are accessed through the {\em ops}
field of the \id{SUNMatrix} structure. To illustrate this point we
show below the implementation of a typical matrix operation from the
generic {\sunmatrix} module, namely \id{SUNMatZero}, which sets all
values of a matrix \id{A} to zero, returning a flag denoting a
successful/failed operation:
%%
%%
\begin{verbatim}
int SUNMatZero(SUNMatrix A)
{
  return((int) A->ops->zero(A));
}
\end{verbatim}
%%
%%
Section \ref{ss:sunmatrix_functions} contains a complete list of all matrix operations
defined by the generic {\sunmatrix} module.

The Fortran 2003 interface provides a \id{bind(C)} derived-type for the
\id{\_generic\_SUNMatrix} and the \id{\_generic\_SUNMatrix\_Ops} structures.
Their definition is given below.
%%
%%
\begin{verbatim}
 type, bind(C), public :: SUNMatrix
  type(C_PTR), public :: content
  type(C_PTR), public :: ops
 end type SUNMatrix

 type, bind(C), public :: SUNMatrix_Ops
  type(C_FUNPTR), public :: getid
  type(C_FUNPTR), public :: clone
  type(C_FUNPTR), public :: destroy
  type(C_FUNPTR), public :: zero
  type(C_FUNPTR), public :: copy
  type(C_FUNPTR), public :: scaleadd
  type(C_FUNPTR), public :: scaleaddi
  type(C_FUNPTR), public :: matvecsetup
  type(C_FUNPTR), public :: matvec
  type(C_FUNPTR), public :: space
 end type SUNMatrix_Ops
\end{verbatim}


%==============================================================================
\subsection{Implementing a custom SUNMatrix}\label{ss:sunmatrix_custom}

A particular implementation of the {\sunmatrix} module must:
\begin{itemize}
\item Specify the {\em content} field of the \id{SUNMatrix} object.
\item Define and implement a minimal subset of the matrix operations.
  See the documentation for each {\sundials} solver to determine which
  {\sunmatrix} operations they require.

  Note that the names of these routines should be unique to that
  implementation in order to permit using more than one {\sunmatrix}
  module (each with different \id{SUNMatrix} internal data
  representations) in the same code.
\item Define and implement user-callable constructor and destructor
  routines to create and free a \id{SUNMatrix} with
  the new {\em content} field and with {\em ops} pointing to the
  new matrix operations.
\item Optionally, define and implement additional user-callable routines
  acting on the newly defined \id{SUNMatrix} (e.g., a routine to print
  the content for debugging purposes).
\item Optionally, provide accessor macros or functions as needed for
  that particular implementation to access different parts
  of the {\em content} field of the newly defined \id{SUNMatrix}.
\end{itemize}

It is recommended that a user-supplied {\sunmatrix} implementation use the
\id{SUNMATRIX\_CUSTOM} identifier.

To aid in the creation of custom {\sunmatrix} modules the generic {\sunmatrix}
module provides two utility functions \id{SUNMatNewEmpty} and
\id{SUNMatVCopyOps}. When used in custom {\sunmatrix} constructors and clone
routines these functions will ease the introduction of any new optional matrix
operations to the {\sunmatrix} API by ensuring only required operations need to
be set and all operations are copied when cloning a matrix. These functions
are desrcribed in Section \ref{ss:sunmatrix_utilities}.
